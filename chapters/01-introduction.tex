\chapter{Introduction}

% 1) paralalicia vseobecne
These days, we are increasingly encountering parallel programs.
A dozen programs that have been written in a typical way for single-core systems cannot take advantage of the presence of computers with multiple cores.
When we wanted to speed up problem-solving, we wanted to create something that would eliminate our time on calculations.
Thus, we invented the computer, which at the beginning knew relatively nothing to do.
However, after a few years, all this changed, and the computer solved problems that took a person many days.
Nowadays, we live in a time when computers have significantly improved execution time by solving different problems using parallelism.

% 2) úvod do problematiky
Several years ago, \emph{Google} released a technology that defined and changed our application deployment and management perspective.
An iterative sequence of small steps caused this revolution (i.e., physical, virtual and container era).
\emph{Kubernetes}~\cite{history, kubernetes, kubernetesBook} is a container management system, and recently in another mini-iteration, brought a new concept of how to organise containers more efficiently \ --- \ the \emph{Operator pattern}.
\emph{Operator pattern} aims to capture how to extend and implement automation tasks beyond \emph{Kubernetes}.
One such Operator is developed and maintained as part of an open-source project called Strimzi~\cite{strimziDoc, strimziBlogPosts}.
The \emph{Strimzi} project brings together several tools (i.e.,
Prometheus\footnote{Prometheus \---\ open-source metrics-based project. Moreover, it provides an alerting system with incredible features, in case of interest \url{https://prometheus.io/}},
Grafana\footnote{Grafana \---\ open-source project, which primary responsibility is to show user interactive visualisation to track crucial parts of the system via great user interface. (\url{https://grafana.com/})},
Jaeger\footnote{Jaeger (Jaeger Tracing) \---\ is a an product, which find and help troubleshoot problems in distributive systems. (\url{https://www.jaegertracing.io/})},
Keycloak\footnote{Keycloak \---\ open-source project for securing applications (authentication and authorization). (\url{https://www.keycloak.org/})}) that take care of the deployment of Apache Kafka~\cite{apacheKafkaDefinitiveGuide, apacheKafkaDesignDistributedSystems, kafkaStreamsBook, kafkaDocumentation} on Kubernetes.
Complexity, horizontal scalability and distribution system;
all these attributes have a system called \emph{Apache Kafka}.
Unfortunately, these attributes make the system an exceedingly complex entity to verify.
Therefore, one of the biggest challenges of using \emph{Kubernetes} is effectively and quickly testing projects like \emph{Kafka} and \emph{Strimzi} while verifying integration with similar products.
Regarding the resources required to deploy \emph{Kafka} on virtual machines or containers, it is relatively simple to compare Kafka's deployment on \emph{Kubernetes}.
Nevertheless, this causes time problems for our \emph{Strimzi} product testing.
To solve this problem, we have adopted the principles of parallel execution and created a mechanism within the Strimzi system tests, which runs tests in parallel against only one cluster of \emph{Kubernetes}.
% 3) related work

Related work focuses on improving the overall verification time of a Strimzi product.
For a long time and many releases of Strimzi, testing using a sequential computational model has been extremely slow.
Furthermore, the product contains about fifteen of the most critical possible combinations of product deployment, each of which lasts over sixty hours.
This sequential computational model is not a recommended candidate for verifying such numerous deployments.
An attentive reader could see the entire test time approaching one thousand hours, which is approximately one and a half month.
\begin{figure}[!ht]
    \centering
    \includegraphics[scale=0.7]{obrazky-figures/01-intro/00-intro-better-one}
    \label{00:fig:evolution}
    \caption{Evolution of our test framework execution}
\end{figure}
Nevertheless, as part of this effort for coarse-grained parallelism in performing multiple product deployments, it partially accelerated the overall computation.
However, this approach is not horizontally scalable due to our cloud services that provide resources (i.e., bare metals, virtual machines, containers).
Therefore, we got to the last opportunity to improve the computation using the vertical scalability of the resources (i.e., memory, central processing units) that the cloud services offer us.
This information motivated us to design and implement a mechanism of fine-grained parallelism in our test framework.
Figure \ref{00:fig:evolution} shows the overall evolution of our test framework and summarises the previously mentioned sentences.
\bigskip
% 4) prínosy tejto práce

\textbf{Acquisition of this work} \quad This work deals with the parallelisation of Kubernetes Strimzi system tests.
The author contributed the given code to the open-sourced project Strimzi available on Github\footnote{Strimzi Github repository \---\ \url{https://github.com/strimzi/strimzi-kafka-operator}}, which also makes it possible to inspire other \emph{kube-native}\footnote{Kube-native \---\ it is a product that has been moved from the standalone world to the Kubernetes world. Moreover, it provides a communication interface (i.e., Kubernets REST API) with which it manages individual components (i.e., Apache Kafka is a standalone application, and Strimzi is a \emph{kube-native} product because it encapsulates Apache Kafka and provides a communication interface for the user.} products to implement such solutions.
The comprehensive benefit of this work is the significant acceleration of the verification process.
\bigskip
% 5) štruktúra práce

\textbf{The structure of the diploma thesis} \quad The author decomposed the whole work into seven chapters together with an introduction.
In Chapter~\ref{02:chapter:title}, the reader learns about the theoretical background to understand the overall thesis (i.e., Kubernetes, Apache Kafka, Strimzi).
Subsequently, we explain the fundamental concepts of parallelism (i.e., Amdahl's law (\ref{04:amdalhlaw}), Shared memory (\ref{04:sharedmemory}), Process and Thread (\ref{04:processesandthreads}), Synchronisation (\ref{04:synchronization}) in Chapter~\ref{03:chapter:title}.
Chapter~\ref{04:chapter:title} presents bottlenecks in the current approach of testing the Strimzi product and proposes a brand-new computational architecture that solves many issues.
Moreover, in Chapter~\ref{05:chapter:title}, we describe the implementation of the proposed architecture.
In the penultimate part of this thesis (Chapter~\ref{06:chapter:title}), we summarise the results from many experiments with deep analysis on the thesis implementation.
Finally, we conclude the entire diploma thesis with the knowledge that has been acquired and at the same time with the possible future work in Chapter \ref{07:chapter:title} and Chapter \ref{08:chapter:title}.