% This file should be replaced with your file with an appendices (headings below are examples only)

% Placing of table of contents of the memory media here should be consulted with a supervisor
%\chapter{Contents of the included storage media}

%\chapter{Manual}

%\chapter{Configuration file}

%\chapter{Scheme of RelaxNG configuration file}

%\chapter{Poster}

%! Author = morsak
%! Date = 13.02.2022

\chapter{Manual}
\label{30:appendix:a}

The author assume \dots by step how to run (a) method-wide parallelisation, (b) class-wide parallelisation.
What is needed blah blah..

\chapter{Implementation details}
\label{30:appendix:b}

\begin{lstlisting}[language=Java,label=resourcemanager:complete:create:method,caption=Complete thead-safe method for parallel creation resources,frame=tb]
@SafeVarargs
public final <T extends HasMetadata> void createResource(
    ExtensionContext testContext,
    boolean waitReady, T... resources) {
    for (T resource : resources) {
        ResourceType<T> type = findResourceType(resource);
        LOGGER.info("Create/Update {} {} in namespace {}",
            resource.getKind(), resource.getMetadata().getName(),
            resource.getMetadata().getNamespace() == null ? "(not set)"
                : resource.getMetadata().getNamespace());

        // ignore test context of shared Cluster Operator
        if (testContext != BeforeAllOnce.getSharedExtensionContext()) {
            // if it is parallel namespace test we are gonna replace
            // resource a namespace
            if (StUtils.isParallelNamespaceTest(testContext)) {
                if (!Environment.isNamespaceRbacScope()) {
                    final String namespace = testContext
                        .getStore(ExtensionContext.Namespace.GLOBAL)
                        .get(Constants.NAMESPACE_KEY).toString();
                    LOGGER.info("Using Namespace: {}", namespace);
                    resource.getMetadata().setNamespace(namespace);
                }
            }
        }

        type.create(resource);

        synchronized (this) {
            STORED_RESOURCES.computeIfAbsent(testContext.getDisplayName(),
                k -> new Stack<>());
            STORED_RESOURCES.get(testContext.getDisplayName()).push(
                new ResourceItem<T>(
                    () -> deleteResource(resource),
                    resource
                ));
        }
    }

    if (waitReady) {
        for (T resource : resources) {
            ResourceType<T> type = findResourceType(resource);
            assertTrue(waitResourceCondition(resource,
                ResourceCondition.readiness(type)),
                String.format("Timed out waiting for %s %s in namespace
                %s to be ready",
                resource.getKind(),
                resource.getMetadata().getName(),
                resource.getMetadata().getNamespace()));
        }
    }
}
\end{lstlisting}

\begin{lstlisting}[language=Java,label=resourcemanager:complete:delete:method,caption=Complete thead-safe method for parallel deletion resources,frame=tb]
public void deleteResources(ExtensionContext testContext) throws Exception {
    LOGGER.info(String.join("", Collections.nCopies(76, "#")));
    if (!STORED_RESOURCES.containsKey(testContext.getDisplayName()) ||
        STORED_RESOURCES.get(testContext.getDisplayName()).isEmpty()) {
        LOGGER.info("In context {} is everything deleted.",
            testContext.getDisplayName());
    } else {
        LOGGER.info("Delete all resources for {}",
            testContext.getDisplayName());
    }

    // if stack is created for specific test suite or test case
    AtomicInteger numberOfResources =
        STORED_RESOURCES.get(testContext.getDisplayName()) != null ?
        new AtomicInteger(STORED_RESOURCES.get(
        testContext.getDisplayName()).size()) :
        // stack has no elements
        new AtomicInteger(0);
    while (STORED_RESOURCES.containsKey(testContext.getDisplayName()) &&
        numberOfResources.get() > 0) {
        STORED_RESOURCES.get(testContext.getDisplayName())
            .parallelStream().parallel().forEach(
            resourceItem -> {
                try {
                    resourceItem.getThrowableRunner().run();
                } catch (Exception e) {
                    e.printStackTrace();
                }
                numberOfResources.decrementAndGet();
            }
        );
    }
    STORED_RESOURCES.remove(testContext.getDisplayName());
    LOGGER.info(String.join("", Collections.nCopies(76, "#")));
}
\end{lstlisting}


\begin{lstlisting}[language=Java,label=resourcemanager:complete:sync:method,caption=Complete thead-safe method for synchronize resources,frame=tb]
public final <T extends HasMetadata> void synchronizeResources(
    ExtensionContext testContext) {
    Stack<ResourceItem> resources = STORED_RESOURCES.get(
        testContext.getDisplayName());

    // sync all resources
    for (ResourceItem resource : resources) {
        if (resource.getResource() == null) {
            continue;
        }
        ResourceType<T> type = findResourceType((T) resource.getResource());

        waitResourceCondition((T) resource.getResource(),
            ResourceCondition.readiness(type));
    }
}
\end{lstlisting}

\begin{lstlisting}[language=Java,label=resourcemanager:supported:resources,caption=List of supported resources inside ResourceManager,frame=tb]
private final ResourceType<?>[] resourceTypes = new ResourceType[]{
    new KafkaBridgeResource(),
    new KafkaClientsResource(),
    new KafkaConnectorResource(),
    new KafkaConnectResource(),
    new KafkaMirrorMaker2Resource(),
    new KafkaMirrorMakerResource(),
    new KafkaRebalanceResource(),
    new KafkaResource(),
    new KafkaTopicResource(),
    new KafkaUserResource(),
    new BundleResource(),
    new ClusterRoleBindingResource(),
    new DeploymentResource(),
    new JobResource(),
    new NetworkPolicyResource(),
    new RoleBindingResource(),
    new ServiceResource(),
    new ConfigMapResource(),
    new ServiceAccountResource(),
    new RoleResource(),
    new ClusterRoleResource(),
    new ClusterOperatorCustomResourceDefinition(),
    new SecretResource(),
    new ValidatingWebhookConfigurationResource()
};
\end{lstlisting}