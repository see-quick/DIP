% This file should be replaced with your file with an appendices (headings below are examples only)

% Placing of table of contents of the memory media here should be consulted with a supervisor
%\chapter{Contents of the included storage media}

%\chapter{Manual}

%\chapter{Configuration file}

%\chapter{Scheme of RelaxNG configuration file}

%\chapter{Poster}

%! Author = morsak
%! Date = 13.02.2022

\chapter{Manual}
\label{30:appendix:a}

The author assume that one has already prepare Kubernetes cluster and it is connected to such instance.
Here is the a few steps how to run (a) method-wide or (b) class-wide parallelisation:
\begin{enumerate}[itemsep=1mm, parsep=0pt]
    \item clone the Strimzi repository \---\ git clone https://github.com/strimzi/strimzi-kafka-operator
    \item enter the cloned repository \---\ cd strimzi-kafka-operator
    \item download needed utilities in directory development-docs/DEV\_GUIDE.md
    \item build Strimzi project \---\ mvn clean install -DskipTests=true -Dmaven.javadoc.skip=true
    \item (a) run system tests using five threads by method-wide parallelisation
    \begin{itemize}
        \item  mvn verify -pl systemtest -Pall -Djunit.jupiter.execution.parallel.enabled=true -Djunit.jupiter.execution.parallel.config.fixed.parallelism=10
    \end{itemize}
    \item (b) run system tests using five threads by class-wide parallelisation
    \begin{itemize}
        \item mvn verify -pl systemtest -Pall -Djunit.jupiter.execution.parallel.enabled=true \\
        -Djunit.jupiter.execution.parallel.config.fixed.parallelism=10 \\
        -Djunit.jupiter.execution.parallel.mode.classes.default=concurrent
    \end{itemize}
    \item (optional) one can also run such parallelisation in the InteliJ IDE by specifying these properties inside configuration.
\end{enumerate}

\chapter{Implementation details}
\label{30:appendix:b}

\begin{lstlisting}[language=Java,label=resourcemanager:complete:create:method,caption=Complete thead-safe method for parallel creation resources,frame=tb]
@SafeVarargs
public final <T extends HasMetadata> void createResource(
    ExtensionContext testContext,
    boolean waitReady, T... resources) {
    for (T resource : resources) {
        ResourceType<T> type = findResourceType(resource);
        LOGGER.info("Create/Update {} {} in namespace {}",
            resource.getKind(), resource.getMetadata().getName(),
            resource.getMetadata().getNamespace() == null ? "(not set)"
                : resource.getMetadata().getNamespace());

        // ignore test context of shared Cluster Operator
        if (testContext != BeforeAllOnce.getSharedExtensionContext()) {
            // if it is parallel namespace test we are gonna replace
            // resource a namespace
            if (StUtils.isParallelNamespaceTest(testContext)) {
                if (!Environment.isNamespaceRbacScope()) {
                    final String namespace = testContext
                        .getStore(ExtensionContext.Namespace.GLOBAL)
                        .get(Constants.NAMESPACE_KEY).toString();
                    LOGGER.info("Using Namespace: {}", namespace);
                    resource.getMetadata().setNamespace(namespace);
                }
            }
        }

        type.create(resource);

        synchronized (this) {
            STORED_RESOURCES.computeIfAbsent(testContext.getDisplayName(),
                k -> new Stack<>());
            STORED_RESOURCES.get(testContext.getDisplayName()).push(
                new ResourceItem<T>(
                    () -> deleteResource(resource),
                    resource
                ));
        }
    }

    if (waitReady) {
        for (T resource : resources) {
            ResourceType<T> type = findResourceType(resource);
            assertTrue(waitResourceCondition(resource,
                ResourceCondition.readiness(type)),
                String.format("Timed out waiting for %s %s in namespace
                %s to be ready",
                resource.getKind(),
                resource.getMetadata().getName(),
                resource.getMetadata().getNamespace()));
        }
    }
}
\end{lstlisting}

\begin{lstlisting}[language=Java,label=resourcemanager:complete:delete:method,caption=Complete thead-safe method for parallel deletion resources,frame=tb]
public void deleteResources(ExtensionContext testContext) throws Exception {
    LOGGER.info(String.join("", Collections.nCopies(76, "#")));
    if (!STORED_RESOURCES.containsKey(testContext.getDisplayName()) ||
        STORED_RESOURCES.get(testContext.getDisplayName()).isEmpty()) {
        LOGGER.info("In context {} is everything deleted.",
            testContext.getDisplayName());
    } else {
        LOGGER.info("Delete all resources for {}",
            testContext.getDisplayName());
    }

    // if stack is created for specific test suite or test case
    AtomicInteger numberOfResources =
        STORED_RESOURCES.get(testContext.getDisplayName()) != null ?
        new AtomicInteger(STORED_RESOURCES.get(
        testContext.getDisplayName()).size()) :
        // stack has no elements
        new AtomicInteger(0);
    while (STORED_RESOURCES.containsKey(testContext.getDisplayName()) &&
        numberOfResources.get() > 0) {
        STORED_RESOURCES.get(testContext.getDisplayName())
            .parallelStream().parallel().forEach(
            resourceItem -> {
                try {
                    resourceItem.getThrowableRunner().run();
                } catch (Exception e) {
                    e.printStackTrace();
                }
                numberOfResources.decrementAndGet();
            }
        );
    }
    STORED_RESOURCES.remove(testContext.getDisplayName());
    LOGGER.info(String.join("", Collections.nCopies(76, "#")));
}
\end{lstlisting}


\begin{lstlisting}[language=Java,label=resourcemanager:complete:sync:method,caption=Complete thead-safe method for synchronize resources,frame=tb]
public final <T extends HasMetadata> void synchronizeResources(
    ExtensionContext testContext) {
    Stack<ResourceItem> resources = STORED_RESOURCES.get(
        testContext.getDisplayName());

    // sync all resources
    for (ResourceItem resource : resources) {
        if (resource.getResource() == null) {
            continue;
        }
        ResourceType<T> type = findResourceType((T) resource.getResource());

        waitResourceCondition((T) resource.getResource(),
            ResourceCondition.readiness(type));
    }
}
\end{lstlisting}

\begin{lstlisting}[language=Java,label=resourcemanager:supported:resources,caption=List of supported resources inside ResourceManager,frame=tb]
private final ResourceType<?>[] resourceTypes = new ResourceType[]{
    new KafkaBridgeResource(),
    new KafkaClientsResource(),
    new KafkaConnectorResource(),
    new KafkaConnectResource(),
    new KafkaMirrorMaker2Resource(),
    new KafkaMirrorMakerResource(),
    new KafkaRebalanceResource(),
    new KafkaResource(),
    new KafkaTopicResource(),
    new KafkaUserResource(),
    new BundleResource(),
    new ClusterRoleBindingResource(),
    new DeploymentResource(),
    new JobResource(),
    new NetworkPolicyResource(),
    new RoleBindingResource(),
    new ServiceResource(),
    new ConfigMapResource(),
    new ServiceAccountResource(),
    new RoleResource(),
    new ClusterRoleResource(),
    new ClusterOperatorCustomResourceDefinition(),
    new SecretResource(),
    new ValidatingWebhookConfigurationResource()
};
\end{lstlisting}

\begin{lstlisting}[language=Java,label=cluster:operator:builder:pattern,caption=Cluster Operator builder pattern,frame=tb]
public class SetupClusterOperator {

    private ExtensionContext extensionContext;
    private String clusterOperatorName;
    private String namespaceInstallTo;
    private String namespaceToWatch;
    private List<String> bindingsNamespaces;
    private long operationTimeout;
    private long reconciliationInterval;
    private List<EnvVar> extraEnvVars;
    private Map<String, String> extraLabels;
    private ClusterOperatorRBACType clusterOperatorRBACType;

    public SetupClusterOperator(SetupClusterOperatorBuilder builder) {
        this.extensionContext = builder.extensionContext;
        this.clusterOperatorName = builder.clusterOperatorName;
        this.namespaceInstallTo = builder.namespaceInstallTo;
        this.namespaceToWatch = builder.namespaceToWatch;
        this.bindingsNamespaces = builder.bindingsNamespaces;
        this.operationTimeout = builder.operationTimeout;
        this.reconciliationInterval = builder.reconciliationInterval;
        this.extraEnvVars = builder.extraEnvVars;
        this.extraLabels = builder.extraLabels;
        this.clusterOperatorRBACType = builder.clusterOperatorRBACType;

        // assign defaults is something is not specified
        if (this.clusterOperatorName == null || this.clusterOperatorName.isEmpty()) {
            this.clusterOperatorName = Constants.STRIMZI_DEPLOYMENT_NAME;
        }
        // if namespace is not set we install operator to 'infra-namespace'
        if (this.namespaceInstallTo == null || this.namespaceInstallTo.isEmpty()) {
            this.namespaceInstallTo = Constants.INFRA_NAMESPACE;
        }
        if (this.namespaceToWatch == null) {
            this.namespaceToWatch = this.namespaceInstallTo;
        }
        if (this.bindingsNamespaces == null) {
            this.bindingsNamespaces = new ArrayList<>();
            this.bindingsNamespaces.add(this.namespaceInstallTo);
        }
        if (this.operationTimeout == 0) {
            this.operationTimeout = Constants.CO_OPERATION_TIMEOUT_DEFAULT;
        }
        if (this.reconciliationInterval == 0) {
            this.reconciliationInterval = Constants.RECONCILIATION_INTERVAL;
        }
        if (this.extraEnvVars == null) {
            this.extraEnvVars = new ArrayList<>();
        }
        if (this.extraLabels == null) {
            this.extraLabels = new HashMap<>();
        }
        if (this.clusterOperatorRBACType == null) {
            this.clusterOperatorRBACType = ClusterOperatorRBACType.CLUSTER;
        }
        instanceHolder = this;
    }

    public static class SetupClusterOperatorBuilder {

        private ExtensionContext extensionContext;
        private String clusterOperatorName;
        private String namespaceInstallTo;
        private String namespaceToWatch;
        private List<String> bindingsNamespaces;
        private long operationTimeout;
        private long reconciliationInterval;
        private List<EnvVar> extraEnvVars;
        private Map<String, String> extraLabels;
        private ClusterOperatorRBACType clusterOperatorRBACType;

        public SetupClusterOperatorBuilder withExtensionContext(
            ExtensionContext extensionContext) {
            this.extensionContext = extensionContext;
            return self();
        }
        public SetupClusterOperatorBuilder withClusterOperatorName(
            String clusterOperatorName) {
            this.clusterOperatorName = clusterOperatorName;
            return self();
        }
        public SetupClusterOperatorBuilder withNamespace(
            String namespaceInstallTo) {
            this.namespaceInstallTo = namespaceInstallTo;
            return self();
        }
        public SetupClusterOperatorBuilder withWatchingNamespaces(
            String namespaceToWatch) {
            this.namespaceToWatch = namespaceToWatch;
            return self();
        }

        public SetupClusterOperatorBuilder withBindingsNamespaces(
            List<String> bindingsNamespaces) {
            this.bindingsNamespaces = bindingsNamespaces;
            return self();
        }

        public SetupClusterOperatorBuilder withOperationTimeout(
            long operationTimeout) {
            this.operationTimeout = operationTimeout;
            return self();
        }
        public SetupClusterOperatorBuilder withReconciliationInterval(
            long reconciliationInterval) {
            this.reconciliationInterval = reconciliationInterval;
            return self();
        }

        // currently supported only for Bundle installation
        public SetupClusterOperatorBuilder withExtraEnvVars(
            List<EnvVar> envVars) {
            this.extraEnvVars = envVars;
            return self();
        }

        // currently supported only for Bundle installation
        public SetupClusterOperatorBuilder withExtraLabels(
            Map<String, String> extraLabels) {
            this.extraLabels = extraLabels;
            return self();
        }

        // currently supported only for Bundle installation
        public SetupClusterOperatorBuilder withClusterOperatorRBACType(
            ClusterOperatorRBACType clusterOperatorRBACType) {
            this.clusterOperatorRBACType = clusterOperatorRBACType;
            return self();
        }

        private SetupClusterOperatorBuilder self() {
            return this;
        }

        public SetupClusterOperator createInstallation() {
            return new SetupClusterOperator(this);
        }
    }
}
\end{lstlisting}