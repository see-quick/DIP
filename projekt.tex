%==============================================================================
% tento soubor pouzijte jako zaklad
% this file should be used as a base for the thesis
% Autoři / Authors: 2008 Michal Bidlo, 2019 Jaroslav Dytrych
% Kontakt pro dotazy a připomínky: sablona@fit.vutbr.cz
% Contact for questions and comments: sablona@fit.vutbr.cz
%==============================================================================
% kodovani: UTF-8 (zmena prikazem iconv, recode nebo cstocs)
% encoding: UTF-8 (you can change it by command iconv, recode or cstocs)
%------------------------------------------------------------------------------
% zpracování / processing: make, make pdf, make clean
%==============================================================================
% Soubory, které je nutné upravit nebo smazat: / Files which have to be edited or deleted:
%   projekt-20-literatura-bibliography.bib - literatura / bibliography
%   projekt-01-kapitoly-chapters.tex - obsah práce / the thesis content
%   projekt-01-kapitoly-chapters-en.tex - obsah práce v angličtině / the thesis content in English
%   projekt-30-prilohy-appendices.tex - přílohy / appendices
%   projekt-30-prilohy-appendices-en.tex - přílohy v angličtině / appendices in English
%==============================================================================
\documentclass[english]{fitthesis} % bez zadání - pro začátek práce, aby nebyl problém s překladem
%\documentclass[english]{fitthesis} % without assignment - for the work start to avoid compilation problem
%\documentclass[zadani]{fitthesis} % odevzdani do wisu a/nebo tisk s barevnými odkazy - odkazy jsou barevné
%\documentclass[english,zadani]{fitthesis} % for submission to the IS FIT and/or print with color links - links are color
%\documentclass[zadani,print]{fitthesis} % pro černobílý tisk - odkazy jsou černé
%\documentclass[english,zadani,print]{fitthesis} % for the black and white print - links are black
%\documentclass[zadani,cprint]{fitthesis} % pro barevný tisk - odkazy jsou černé, znak VUT barevný
%\documentclass[english,zadani,cprint]{fitthesis} % for the print - links are black, logo is color
% * Je-li práce psaná v anglickém jazyce, je zapotřebí u třídy použít 
%   parametr english následovně:
%   If thesis is written in English, it is necessary to use 
%   parameter english as follows:
%      \documentclass[english]{fitthesis}
% * Je-li práce psaná ve slovenském jazyce, je zapotřebí u třídy použít 
%   parametr slovak následovně:
%   If the work is written in the Slovak language, it is necessary 
%   to use parameter slovak as follows:
%      \documentclass[slovak]{fitthesis}
% * Je-li práce psaná v anglickém jazyce se slovenským abstraktem apod., 
%   je zapotřebí u třídy použít parametry english a enslovak následovně:
%   If the work is written in English with the Slovak abstract, etc., 
%   it is necessary to use parameters english and enslovak as follows:
%      \documentclass[english,enslovak]{fitthesis}

% Základní balíčky jsou dole v souboru šablony fitthesis.cls
% Basic packages are at the bottom of template file fitthesis.cls
% zde můžeme vložit vlastní balíčky / you can place own packages here

% Kompilace po částech (rychlejší, ale v náhledu nemusí být vše aktuální)
% Compilation piecewise (faster, but not all parts in preview will be up-to-date)
% \usepackage{subfiles}

% Nastavení cesty k obrázkům
% Setting of a path to the pictures
%\graphicspath{{obrazky-figures/}{./obrazky-figures/}}
%\graphicspath{{obrazky-figures/}{../obrazky-figures/}}

%---rm---------------
\renewcommand{\rmdefault}{lmr}%zavede Latin Modern Roman jako rm / set Latin Modern Roman as rm
%---sf---------------
\renewcommand{\sfdefault}{qhv}%zavede TeX Gyre Heros jako sf
%---tt------------
\renewcommand{\ttdefault}{lmtt}% zavede Latin Modern tt jako tt

% vypne funkci šablony, která automaticky nahrazuje uvozovky,
% aby nebyly prováděny nevhodné náhrady v popisech API apod.
% disables function of the template which replaces quotation marks
% to avoid unnecessary replacements in the API descriptions etc.
\csdoublequotesoff



\usepackage{url}


% =======================================================================
% balíček "hyperref" vytváří klikací odkazy v pdf, pokud tedy použijeme pdflatex
% problém je, že balíček hyperref musí být uveden jako poslední, takže nemůže
% být v šabloně
% "hyperref" package create clickable links in pdf if you are using pdflatex.
% Problem is that this package have to be introduced as the last one so it 
% can not be placed in the template file.
\ifWis
\ifx\pdfoutput\undefined % nejedeme pod pdflatexem / we are not using pdflatex
\else
  \usepackage{color}
  \usepackage[unicode,colorlinks,hyperindex,plainpages=false,pdftex]{hyperref}
  \definecolor{hrcolor-ref}{RGB}{223,52,30}
  \definecolor{hrcolor-cite}{HTML}{2F8F00}
  \definecolor{hrcolor-urls}{HTML}{092EAB}
  \hypersetup{
	linkcolor=hrcolor-ref,
	citecolor=hrcolor-cite,
	filecolor=magenta,
	urlcolor=hrcolor-urls
  }
  \def\pdfBorderAttrs{/Border [0 0 0] }  % bez okrajů kolem odkazů / without margins around links
  \pdfcompresslevel=9
\fi
\else % pro tisk budou odkazy, na které se dá klikat, černé / for the print clickable links will be black
\ifx\pdfoutput\undefined % nejedeme pod pdflatexem / we are not using pdflatex
\else
  \usepackage{color}
  \usepackage[unicode,colorlinks,hyperindex,plainpages=false,pdftex,urlcolor=black,linkcolor=black,citecolor=black]{hyperref}
  \definecolor{links}{rgb}{0,0,0}
  \definecolor{anchors}{rgb}{0,0,0}
  \def\AnchorColor{anchors}
  \def\LinkColor{links}
  \def\pdfBorderAttrs{/Border [0 0 0] } % bez okrajů kolem odkazů / without margins around links
  \pdfcompresslevel=9
\fi
\fi
% Řešení problému, kdy klikací odkazy na obrázky vedou za obrázek
% This solves the problems with links which leads after the picture
\usepackage[all]{hypcap}

% Informace o práci/projektu / Information about the thesis
%---------------------------------------------------------------------------
\projectinfo{
  %Prace / Thesis
  project={DP},            %typ práce BP/SP/DP/DR  / thesis type (SP = term project)
  year={2022},             % rok odevzdání / year of submission
  date=\today,             % datum odevzdání / submission date
  %Nazev prace / thesis title
  title.cs={Konfigurateľná paralelná exekúcia systémových testov v rámci projektu Strimzi},  % název práce v češtině či slovenštině (dle zadání) / thesis title in czech language (according to assignment)
  title.en={Configurable parallel execution of system tests within the Strimzi project}, % název práce v angličtině / thesis title in english
  %title.length={14.5cm}, % nastavení délky bloku s titulkem pro úpravu zalomení řádku (lze definovat zde nebo níže) / setting the length of a block with a thesis title for adjusting a line break (can be defined here or below)
  %sectitle.length={14.5cm}, % nastavení délky bloku s druhým titulkem pro úpravu zalomení řádku (lze definovat zde nebo níže) / setting the length of a block with a second thesis title for adjusting a line break (can be defined here or below)
  %dectitle.length={14.5cm}, % nastavení délky bloku s titulkem nad prohlášením pro úpravu zalomení řádku (lze definovat zde nebo níže) / setting the length of a block with a thesis title above declaration for adjusting a line break (can be defined here or below)
  %Autor / Author
  author.name={Maroš},   % jméno autora / author name
  author.surname={Orsák},   % příjmení autora / author surname 
  author.title.p={Bc.}, % titul před jménem (nepovinné) / title before the name (optional)
  %author.title.a={Ph.D.}, % titul za jménem (nepovinné) / title after the name (optional)
  %Ustav / Department
  department={UITS}, % doplňte příslušnou zkratku dle ústavu na zadání: UPSY/UIFS/UITS/UPGM / fill in appropriate abbreviation of the department according to assignment: UPSY/UIFS/UITS/UPGM
  % Školitel / supervisor
  supervisor.name={Milan},   % jméno školitele / supervisor name 
  supervisor.surname={Češka},   % příjmení školitele / supervisor surname
  supervisor.title.p={RNDr.},   %titul před jménem (nepovinné) / title before the name (optional)
  supervisor.title.a={Ph.D.},    %titul za jménem (nepovinné) / title after the name (optional)
  % Klíčová slova / keywords
  keywords.cs={Sem budou zapsána jednotlivá klíčová slova v českém (slovenském) jazyce, oddělená čárkami.}, % klíčová slova v českém či slovenském jazyce / keywords in czech or slovak language
  % Abstrakt / Abstract
  abstract.cs={Do tohoto odstavce bude zapsán výtah (abstrakt) práce v českém (slovenském) jazyce.}, % abstrakt v českém či slovenském jazyce / abstract in czech or slovak language
  abstract.en={
In recent years, many companies adopted the Kubernetes system and architecture of Microservices. This technology opened up a lot of new possibilities not just for big companies, but also for small users of the containers. Since Kubernetes is a container-orchestration system a new idea about how to orchestrate the containers came \---\ the Operator pattern. One of these operators is developed and maintained under an open-source project called Strimzi. This project gathers together several tools, which takes care of the deployment of Apache Kafka on top of Kubernetes.
Since Kafka is a vast, horizontally scalable, and distributed system, you can imagine that the installation of such a system is a relatively complex action. Therefore, one of the biggest challenges of the Kubernetes environment is how to effectively and quickly test projects such as Kafka and Strimzi and at the same time verify integration with other similar products. 
Additionally, the resources needed by Kubernetes are much more demanding and thus rank among the most complex. This means that need for the resources is higher compared to the deployment of Kafka on virtual machines or typical container instances. To tackle this problem, we adopt the principles of parallel execution and created a mechanism within Strimzi system tests, which runs tests in parallel against only a single Kubernetes cluster. Furthermore, we proposed a brand new architecture of the end-to-end tests. The improvements aim at \textit{scalability} and \textit{reduction of execution time}. Through several experiments, this paper shows that proposed mechanism with different configurations of Kubernetes cluster (including \textit{number of Kubernetes nodes}, \textit{number of tests and suites executed in parallel}) significantly accelerated execution of the tests.
  
  
  %In recent years, more and more companies adopted the %Kubernetes system and architecture of microservices. This %technology opened up new possibilities and with it came the  %operator pattern. One of the operators is also called Strimzi.
  %
  %This approach introduced a completely new pattern usable in %cloud world - Operators. One of this operators is also %Strimzi. This project collects together several tools, which %takes care about Apache Kafka deployed on top of Kubernetes. %Since Kafka is very huge, robust and easilly scalable system, %you can imagine that deployment of systems like this is not %trivial. One of the biggest challenges in this area is %effective testing of systems like Strimzi, because you combine %there together multiple huge applications. Moreover, the %resources which you need to setup Kubernetes cluster are much %more bigger than for deploy for instance Kafka on some virtual %machine. With this knowledge we can ask - how we can %effectively test operators such as Strimzi and how to avoid %false positive scenarios when several tests are running in %parallel? Those and more questions will be ansvered in this %thesis, where we will focus on current implementaton of %Strimzi system testing and how to improve it to effectively %test it in parallel with multiple operators and Kafka clusters %deployed into one Kubernetes cluster.
  
  % My version
  % Strimzi is a collection of Kubernetes operators for Apache Kafka based on the idea % to be as easily configurable as possible. Since Kafka is a distributed, % horizontally scalable middleware product, the testing itself could take much time. % Setting up Kafka cluster with other resources is time-consuming, not even speaking % about deploying Strimzi operators next to it. The current approach of testing the % Strimzi product is sluggish and running regression profile takes more than thirty % hours by using sequence execution. To tackle this problem, we adopt the principles % of parallel execution and create a mechanism, which runs end-to-end tests in % parallel against only a single Kubernetes cluster. Moreover, we proposed a brand % new architecture of the end-to-end tests. The improvements aim at % \textit{scalability} and \textit{reduction of execution time}. Through several % experiments, this paper shows that proposed mechanism with different configurations % of Kubernetes cluster (including \textit{number of Kubernetes nodes}, % \textit{number of tests or suites executed in parallel}) significantly accelerated % execution of the tests. 
  }, % klíčová slova v českém či slovenském jazyce / keywords in czech or slovak language
  keywords.en={Strimzi, Kubernetes, Orchestration, Clustering, Azure, Openstack, AWS, Apache Kafka, Distributed systems, middleware, end-to-end tests, paralelism, multi-threaded execution, race condition, synchronization, scalability, operators}, % abstrakt v anglickém jazyce / abstract in english
  %abstract.en={An abstract of the work in English will be written in this paragraph.},
  % Prohlášení (u anglicky psané práce anglicky, u slovensky psané práce slovensky) / Declaration (for thesis in english should be in english)
  declaration={Prohlašuji, že jsem tuto bakalářskou práci vypracoval samostatně pod vedením pana X...
Další informace mi poskytli...
Uvedl jsem všechny literární prameny, publikace a další zdroje, ze kterých jsem čerpal.},
  %declaration={I hereby declare that this Bachelor's thesis was prepared as an original work by the author under the supervision of Mr. X
% The supplementary information was provided by Mr. Y
% I have listed all the literary sources, publications and other sources, which were used during the preparation of this thesis.},
  % Poděkování (nepovinné, nejlépe v jazyce práce) / Acknowledgement (optional, ideally in the language of the thesis)
  acknowledgment={V této sekci je možno uvést poděkování vedoucímu práce a těm, kteří poskytli odbornou pomoc
(externí zadavatel, konzultant apod.).},
  %acknowledgment={Here it is possible to express thanks to the supervisor and to the people which provided professional help
%(external submitter, consultant, etc.).},
  % Rozšířený abstrakt (cca 3 normostrany) - lze definovat zde nebo níže / Extended abstract (approximately 3 standard pages) - can be defined here or below
  extendedabstract={Do tohoto odstavce bude zapsán rozšířený výtah (abstrakt) práce v českém (slovenském) jazyce.},
  extabstract.odd={true}, % Začít rozšířený abstrakt na liché stránce? / Should extended abstract start on the odd page?
  %faculty={FIT}, % FIT/FEKT/FSI/FA/FCH/FP/FAST/FAVU/USI/DEF
  faculty.cs={Fakulta informačních technologií}, % Fakulta v češtině - pro využití této položky výše zvolte fakultu DEF / Faculty in Czech - for use of this entry select DEF above
  faculty.en={Faculty of Information Technology}, % Fakulta v angličtině - pro využití této položky výše zvolte fakultu DEF / Faculty in English - for use of this entry select DEF above
  department.cs={Ústav matematiky}, % Ústav v češtině - pro využití této položky výše zvolte ústav DEF nebo jej zakomentujte / Department in Czech - for use of this entry select DEF above or comment it out
  department.en={Institute of Mathematics} % Ústav v angličtině - pro využití této položky výše zvolte ústav DEF nebo jej zakomentujte / Department in English - for use of this entry select DEF above or comment it out
}

% Rozšířený abstrakt (cca 3 normostrany) - lze definovat zde nebo výše / Extended abstract (approximately 3 standard pages) - can be defined here or above
%\extendedabstract{Do tohoto odstavce bude zapsán výtah (abstrakt) práce v českém (slovenském) jazyce.}
% Začít rozšířený abstrakt na liché stránce? / Should extended abstract start on the odd page?
%\extabstractodd{true}

% nastavení délky bloku s titulkem pro úpravu zalomení řádku - lze definovat zde nebo výše / setting the length of a block with a thesis title for adjusting a line break - can be defined here or above
%\titlelength{14.5cm}
% nastavení délky bloku s druhým titulkem pro úpravu zalomení řádku - lze definovat zde nebo výše / setting the length of a block with a second thesis title for adjusting a line break - can be defined here or above
%\sectitlelength{14.5cm}
% nastavení délky bloku s titulkem nad prohlášením pro úpravu zalomení řádku - lze definovat zde nebo výše / setting the length of a block with a thesis title above declaration for adjusting a line break - can be defined here or above
%\dectitlelength{14.5cm}

% řeší první/poslední řádek odstavce na předchozí/následující stránce
% solves first/last row of the paragraph on the previous/next page
\clubpenalty=10000
\widowpenalty=10000

% checklist
\newlist{checklist}{itemize}{1}
\setlist[checklist]{label=$\square$}

% Nechcete-li, aby se u oboustranného tisku roztahovaly mezery pro zaplnění stránky, odkomentujte následující řádek / If you do not want enlarged spacing for filling of the pages in case of duplex printing, uncomment the following line
% \raggedbottom

\begin{document}
  % Vysazeni titulnich stran / Typesetting of the title pages
  % ----------------------------------------------
  \maketitle
  % Obsah
  % ----------------------------------------------
  \setlength{\parskip}{0pt}

  {\hypersetup{hidelinks}\tableofcontents}
  
  % Seznam obrazku a tabulek (pokud prace obsahuje velke mnozstvi obrazku, tak se to hodi)
  % List of figures and list of tables (if the thesis contains a lot of pictures, it is good)
  \ifczech
    \renewcommand\listfigurename{Seznam obrázků}
  \fi
  \ifslovak
    \renewcommand\listfigurename{Zoznam obrázkov}
  \fi
  % {\hypersetup{hidelinks}\listoffigures}
  
  \ifczech
    \renewcommand\listtablename{Seznam tabulek}
  \fi
  \ifslovak
    \renewcommand\listtablename{Zoznam tabuliek}
  \fi
  % {\hypersetup{hidelinks}\listoftables}

  \ifODSAZ
    \setlength{\parskip}{0.5\bigskipamount}
  \else
    \setlength{\parskip}{0pt}
  \fi

  % vynechani stranky v oboustrannem rezimu
  % Skip the page in the two-sided mode
  \iftwoside
    \cleardoublepage
  \fi

  % Text prace / Thesis text
  % ----------------------------------------------
  \ifenglish
    \chapter{Introduction}

% 1) paralalicia vseobecne
These days, we are increasingly encountering parallel programs.
A dozen programs that have been written in a typical way for single-core systems cannot take advantage of the presence of computers with multiple cores.
When we wanted to speed up problem-solving, we wanted to create something that would eliminate our time on calculations.
Thus, we invented the computer, which at the beginning knew relatively nothing to do.
However, after a few years, all this changed, and the computer solved problems that took a person many days.
Nowadays, we live in a time when computers have significantly improved execution time by solving different problems using parallelism.

% 2) úvod do problematiky
In recent years, many companies have adopted Kubernetes (\ref{01:sec:title}) and the microservices architecture it enables.
This technology was opened up many new possibilities not just for large companies but also for small software developers.
Kubernetes is a container-orchestration system, and recently a new concept has emerged around how to orchestrate the containers more efficiently \---\ the Operator pattern.
One such operator is developed and maintained under an open-source project called Strimzi (\ref{03:title}).
The Strimzi project gathers together several tools that take care of Apache Kafka's (\ref{02:sec:title}) deployment on Kubernetes.
Since Kafka is a complex, horizontally scalable, distributed system, one can imagine that its installation is a relatively complex action.
Therefore, one of the biggest challenges of using Kubernetes is how to effectively and quickly test projects such as Kafka and Strimzi and at the same time verify integration with other similar products.
The resources needed by Kubernetes are much more demanding than Kafka's deployment on virtual machines or typical container instances.
To tackle this problem, we adopted the principles of parallel execution and created a mechanism within Strimzi system tests (\ref{02:sec:strimzisystemtests}), which runs tests in parallel against only a single Kubernetes cluster.
Furthermore, we proposed a brand-new architecture for the end-to-end tests (\ref{04:chapter:title}).
The improvements aim at \textit{scalability} and \textit{reduction of execution time}.
Through several experiments (\ref{06:chapter:title}), this paper shows that proposed mechanism with different configurations of the Kubernetes cluster (including \textit{number of Kubernetes nodes}, \textit{number of tests and suites executed in parallel}) significantly accelerated the execution of the tests.

% 3) related work
Related work focuses on improving the overall verification time of a Strimzi product.
For a long time and many releases of Strimzi, testing using a sequential computational model has been extremely slow.
Furthermore, the product contains about fifteen of the most critical possible combinations of product deployment, each of which lasts over sixty hours.
This sequential computational model is not a recommended candidate for verifying such numerous deployments.
An attentive reader could see the entire test time approaching one thousand hours, which is approximately one and a half month.
\begin{figure}[!ht]
    \centering
    \includegraphics[scale=0.7]{obrazky-figures/01-intro/00-intro-better-one}
    \label{00:fig:evolution}
    \caption{Evolution of our test framework execution}
\end{figure}
Nevertheless, as part of this effort for coarse-grained parallelism in performing multiple product deployments, it partially accelerated the overall computation.
However, this approach is not horizontally scalable due to our cloud services that provide resources (i.e., bare metals, virtual machines, containers).
Therefore, we got to the last opportunity to improve the computation using the vertical scalability of the resources (i.e., memory, central processing units) that the cloud services offer us.
This information motivated us to design and implement a mechanism of fine-grained parallelism in our test framework.
Figure \ref{00:fig:evolution} shows the overall evolution of our test framework and summarises the previously mentioned sentences.
\bigskip
% 4) prínosy tejto práce

\textbf{Acquisition of this work} \quad This work deals with the parallelisation of Kubernetes Strimzi system tests.
The author contributed the given code to the open-sourced project Strimzi available on Github\footnote{\textbf{Strimzi Github repository} \---\ \url{https://github.com/strimzi/strimzi-kafka-operator}}, which also makes it possible to inspire other \emph{kube-native}\footnote{\textbf{Kube-native} \---\ it is a product that has been moved from the standalone world to the Kubernetes world. Moreover, it provides a communication interface (i.e., Kubernets REST API) with which it manages individual components (i.e., Apache Kafka is a standalone application, and Strimzi is a \emph{kube-native} product because it encapsulates Apache Kafka and provides a communication interface for the user.} products to implement such solutions.
The comprehensive benefit of this work is the significant acceleration of the verification process.
\bigskip
% 5) štruktúra práce

\textbf{The structure of the diploma thesis} \quad In Chapter \ref{02:chapter:title}, the reader will learn the theoretical background to understand the overall thesis (i.e., Kubernetes (\ref{01:sec:title}), Apache Kafka (\ref{02:sec:title}), Strimzi (\ref{03:title})).
In Chapter \ref{03:chapter:title}, we explain the fundamental concepts of parallelism (i.e., Amdahl's law (\ref{04:amdalhlaw}), Shared memory ()\ref{04:sharedmemory}), Process and Thread (\ref{04:processesandthreads}), Synchronisation (\ref{04:synchronization}).
Chapter \ref{04:chapter:title} presents bottlenecks in the current approach of testing the Strimzi product and proposes a brand-new computational architecture that solves many issues.
In Chapter \ref{05:chapter:title}, we describe the implementation of the proposed architecture.
Chapter \ref{06:chapter:title} presents many experiments with deep analysis on the thesis implementation.
Finally, in Chapters \ref{07:chapter:title},\ref{08:chapter:title}, we conclude the entire diploma with the knowledge that has been acquired, and at the same time, we discuss possible directions for possible future work.

    \chapter{Preliminaries}
\label{02:chapter:title}

This chapter provides the fundamentals of the technologies used across the whole thesis.
Notable technologies used are Kubernetes\footnote{Kubernetes \---\ orchestration system created in 2014 by Google (\url{https://kubernetes.io/})}, Apache Kafka\footnote{Apache Kafka \---\ distributed messaging system originally created in 2013 by LinkedIn (\url{https://kafka.apache.org/})} and Strimzi\footnote{Strimzi \---\ collection of operators for deploying and managing Apache Kafka on top of the Kubernetes (\url{https://strimzi.io/})}, which are described in details in the following sections.
Note that high level descriptions of these technologies were already published in bachelor's thesis~\cite{02-bachelor-thesis} written by the same author as this thesis.
In this chapter the author aims to explain the technology in more technical depth.
Furthermore, some ideas related to Kubernetes were taken from the \emph{The Kubernetes book}~\cite{kubernetesBook}.
Section~\ref{02:sec:strimzisystemtests} describes the \emph{e2e (end-to-end)}  Strimzi tests that run on the top of the Kubernetes cluster.
Also note that author described this topic in series of blogs posts \emph{Introduction to system tests}~\cite{02-blogpost-introduction-to-systemtest} and \emph{How system tests work}~\cite{02-blogpost-how-systemtest-work}.
    \section{Kubernetes}
\label{01:sec:title}

In 2014, Google came with a new concept of container management.
This concept has opened the door for many products to simplify their management of applications deployments.
This technology defined a set of primitives, which collectively provide mechanisms that deploy, maintain and scale applications based on CPU, memory, or custom metrics.
Moreover, it does not create a virtual machine but uses the kernel of the physical computer.
Also known as the lightweight approach compared to virtual machines.
Kubernetes follows leader and follow architecture.
The leader node controls Kubernetes resources and the follow node is responsible for resource creation.
The definition of these resources is given in a declarative way using YAML\footnote{YAML
\---\ human-readable serialization format (\url{https://yaml.org/})} formatted files.

\subsection{History}
\label{fig:history}

So far, we have developed four approaches to managing applications on the top of the operating system~\cite{history}.
In each direction, we have eliminated certain disadvantages based on empirical knowledge.

\begin{enumerate}
    \item \textbf{Running a physical machine}\,---\,The first phase of how to deploy applications was to execute the program on the physical computer.
    This approach was not as practical as it seemed at first.
    The main issues were scalability, management of hardware, security, and price.
    Besides that, sharing memory between five running applications in an identical environment is not ideal.
    Moreover, to isolate the applications from one another, one has to buy five physical servers, significantly increasing costs.

    \item \textbf{Virtualization}\,---\,The next phase has solved problems like scalability, security, and also price.
    This allows an application to run on a single machine without sharing memory, which means it is isolated and encapsulated from other applications.
    Furthermore, you can run many of these virtual machines on a single physical server, and your only limitations are the server resources.
    These virtual machines are independent of each other, and therefore the security is much higher.
    However, resource consumption is still high since each virtual machine includes a full operating system.
    At the same time, the management of these entities is not easy if we imagine production with hundred virtual machines.
    Another limitation is, that sometimes applications need to share information with each other and the strong isolation of VMs makes this difficult.

    \item \textbf{Containerization}\,---\,In the last phase, containerization is considered as a lightweight alternative to virtualization.
    The difference between these two phases is that virtualization is using hypervisors\footnote{Hypervisor - It is a software that manages virtual machines, for instance, VMware or VirtualBox.} to manage all the virtual machines which have operating systems. The container shares the operating system with the server. Similar to virtualization, they have their filesystem, memory, and space. Containerization has become the most popular technology due to the several benefits it offers:
    \begin{itemize}[itemsep=1mm, parsep=0pt]
        \item \textbf{Isolation} \---\ predictable application performance,
        \item \textbf{Observability} \---\ gathering of information, providing metrics, logs,
        \item \textbf{Portability} of distribution in the cloud and OS \---\ runs on basically all available OS, public clouds, and so on,
        \item \textbf{Agile approach} \---\ easy to create and manage smaller container images instead of using virtual machine images, which are usually much larger.
    \end{itemize}
    Unfortunately, containerization still has several shortcomings, such as managing more running containers simultaneously, making debugging challenging, etc.

    \begin{figure}[!ht]
        \centering
        \includegraphics[scale=0.9]{obrazky-figures/02-preliminaries/01-kubernetes/01-timeline-with-picture}
        \label{01:fig:timeline}
        \caption{Evolution of virtual technologies}
    \end{figure}

    \item \textbf{Container orchestration}\,---\,The phase of the present.
    Let's imagine a situation where we run several containers and want to know the container's current state or metadata information.
    It is not straightforward to get such information because we have to look at each running container separately and analyze it.
    Kubernetes brings us a solution to this problem.
    While in containers, we have to search each one individually, so in Kubernetes, we all have it simultaneously.
    Figure \ref{01:fig:timeline} illustrates and summarises the phases of managing an application on top of the operating system, starting in 1950 when the first computer, ENIAC, was assembled—moving to the virtualization era, which started in the early 70s.
    IBM Cambridge Scientific Center began the development of CP-40, the first operating system that implemented complete virtualization.
    However, what is very important to note is that the first known virtualization software was VMware, created in 1997. Afterwards, the lightweight era comes with an idea whose functionality was based on containerization~\cite{docker}.
    Finally, we have a manager who takes care of the overall management of the individual containers and guarantees their reliability, scales it effectively and more. This is what we call a container orchestration system~\cite{kubernetes}.
    It has the following properties:

    \begin{enumerate}[itemsep=1mm, parsep=0pt]
        \item Deployment, StatefulSet, ReplicaSet, and Custom resource definitions (CRDs).
        \item Service and Load balancing (Service discovery).
        \item Storage (Storage orchestration).
        \item Secrets (Secret and configuration management).
    \end{enumerate}
\end{enumerate}

\subsection{Essential components of Kubernetes}
The Linux hosts can be virtual machines, bare-metal servers in the data center, or private or public cloud instances. Production environments typically have more than one master node running because of the need for High Availability\footnote{High availability (HA) \---\ is the characteristic of the system to run without failing for some period of time.}).
\begin{figure}[!h]
    \centering
    \includegraphics[scale=0.82]{obrazky-figures/02-preliminaries/01-kubernetes/02-architecture-master-sketch}
    \caption{Representation of the Master node}
    \label{02:fig:masterNode}
\end{figure}
Kubernetes services from the biggest cloud providers such as Azure Kubernetes Service (AKS), Amazon Elastic Kubernetes Service (EKS), and Google Kubernetes Service (GKS) have five master nodes, which are replicated in case of any failure.
The master node \ref{02:fig:masterNode} contains several components such as \emph{kube-controller-manager}, \emph{kube-scheduler} and \emph{kube-apiserver}.
These components are also called the "control plane".
The \emph{kube-controller-manager} takes care of all controllers where each of these controllers runs as a separate process.
The \emph{Node controller's} responsibility is to control and respond to the current status of the node.
In other words, do a health check of nodes.
There is also the \emph{Endpoint controller} for Service and Pod objects, \emph{Job controller} for Job objects, etc.
All these controllers follow algorithm \ref{01:alg:controllerAlg}.
\begin{algorithm}[H]
    \caption{Generic algorithm for each Kubernetes controller}
    \label{01:alg:controllerAlg}

    \begin{algorithmic}[1]
        \State $desired\_state \gets controller.obtain\_desired\_state()$
        \While{$True$}
            \State $desired\_state \gets controller.obtain\_desired\_state()$
            \State $current\_state\ \gets controller.observe\_current\_state()$

            \If{$current\_state \not\models desired\_state$}
                \State $controller.reconcile()$
            \EndIf

            \State $desired\_state \gets current\_state$
        \EndWhile
    \end{algorithmic}
\end{algorithm}
The \emph{kube-apiserver} works like the controller of API calls and communicates with the \emph{kube-scheduler}.
It makes sure that every created Pod is assigned to run there.
It is worthwhile to mention that we also have a component called \emph{etcd}, which works as a backup for cluster data.
Slave node components~\ref{02:fig:slaveNode} suas \emph{kubelet} have taken care of containers running inside the Pod. \emph{Kube-proxy}, which reflects all the services defined in the kube-apiserver.
In the following Figure~\ref{02:fig:masterAndSlaveNode}, one can see relation between master and slave nodes.

\begin{figure}[!h]
    \centering
    \includegraphics[scale=0.82]{obrazky-figures/02-preliminaries/01-kubernetes/02-architecture-slave-sketch}
    \caption{Representation of the Slave node}
    \label{02:fig:slaveNode}
\end{figure}


\begin{figure}[!h]
    \centering
    \includegraphics[scale=0.92]{obrazky-figures/02-preliminaries/01-kubernetes/02-final-architecture-master-slave}
    \caption{Relation between master and slave nodes}
    \label{02:fig:masterAndSlaveNode}
\end{figure}

\subsection{Common objects}
\label{objects}

\begin{enumerate}[itemsep=1mm, parsep=0pt]
    \item \textbf{Pod} \---\ is the atomic unit of Kubernetes.
    For instance, in the VMware environment, the atomic unit is a virtual machine, and Docker it is a container.
    The term Pod originated from the Docker logo.
    If we think about it, Docker has one whale on his logo, and we call a group of such whales Pod or, in other words, Pod of whales.
    Deductively, we can find out the property of the Pod, that is, that one or more containers can run in it.
    These containers share storage, network, and specification of how to run the container.
    If the container wants to communicate with the other container, this can be achieved using the localhost interface.
    One of the disadvantages of these resources is their lifecycle.
    If Pod crashes or is deleted, it will no longer be possible to copy this Pod.
    Instead, Kubernetes will create a new Pod with a unique ID and a new IP address assigned.

    \item \textbf{Service} \---\ represents how particular components communicate.
    Services provide reliable networking for a set of Pods.
    If Pod fails and Kubernetes creates a new Pod, its IP address is changed.
    Moreover, operations like scaling up or scaling down do the same.
    This is where Services come into play.
    They provide reliable names/alias and IP addresses.
    Furthermore, the Kubernetes service has its DNS name and port.
    It is a stable network abstraction, which provides TCP and UDP load-balancing across a dynamic set of Pods.
    By default, a service in Kubernetes has a type of \emph{ClusterIP}, which means that communication can be established only inside of the Kubernetes cluster.
    The way one can expose an application outside of the cluster is to use the following type of service which Kubernetes offers:
    \begin{itemize}[itemsep=1mm, parsep=0pt]
        \item \textbf{nodeport} \---\ exposes the service to be accessible via node IP with a specific port.
        For instance, you want to expose your HTTP server to be publicly accessible on a specific port.
        \item \textbf{load balancer} \---\  exposes the service externally using a cloud provider's load balance.
        The load balancer is shown in the definition. \emph{.status.loadBalancer} field, where you can find a real IP address.
        For example, if your demands are high and you want an application that requires more ports on specific IPs, then the usage of load balance is a wise choice.
        \item \textbf{ingress} \---\  the previously mentioned types of how to expose a service were service types, but ingress is an entry point for the cluster. \textit{It lets you consolidate your routing rules into a single resource as it can expose multiple services under the same IP address} \cite{ingress}.
    \end{itemize}

    \item \textbf{Namespace} \---\ this concept of namespaces was introduced in order to run numerous virtual clusters inside physical one. \emph{It is great for applying different quotas and access control policies. On the other hand, it is not suitable for strong workload isolation.} By default, Kubernetes starts with three initial namespaces:

    \begin{itemize}
        \item \textbf{default} \---\ the objects which do not have another namespace belongs to the default namespace,
        \item \textbf{Kube-system} \---\ namespace for objects created by the Kubernetes system, i.e. Pods, Kube-proxy, Kube-DNS. Furthermore, the service account in this namespace is used to run the Kubernetes controllers.
        \item \textbf{Kube-public} \---\ \textit{this namespace is created automatically and is recognizable by all users (including those not authenticated). In other words, there is a situation we need to have shared resources across the whole cluster; then we have to make sure that these resources are inside this namespace} \cite{namespaceTypes}
    \end{itemize}

    \item \textbf{Volume} \---\ is data storage.
    The Volume is a separated object, which binds to a Pod.
    The main ideas behind volumes are: at first, assume a scenario when your Pod crashed, and the application will lose all its data, and one would like to retrieve it secondly if one wants to share the same data between more Pods.
    The answer to these problems is the \emph{Kubernetes Volume abstraction}.
\end{enumerate}

\subsection{Controllers}

\begin{enumerate}
    \item \textbf{ReplicaSet} \---\ is the controller that is responsible for the correct number of running Pods.
    Furthermore, ReplicaSet plays a significant role in the Deployment controller, supplying a self-healing mechanism and scale operations.
    The self-healing mechanism guarantees that the Pod is running, and in the event of any error or termination of the Pod, a new one will be created immediately.
    Scale operations guarantee an easy way to increase the number of Application Pods if necessary in the event of a heavy load.
    The same applies even if the given number of Pods is already high (then we use a scale-down operation).
    ReplicaSet also has responsibility for the Rolling Update and Rollback operations available to Deployment.

    \item \textbf{Deployment} \---\ it is one of the most widely used application management controllers in the Kubernetes environment.

    \begin{figure}[!htb]
        \centering
        \includegraphics[scale=1.2]{obrazky-figures/02-preliminaries/01-kubernetes/03-deplyoment-archite}
        \caption{Hierarchy of Deployment, ReplicaSet and Pod inspired by The Kubernetes Book \cite{kubernetesBook}}
        \label{fig:kubernetes:deploymentReplicaSetPod}
    \end{figure}
    Based on our knowledge, the skillful reader will realize that Pod as an atomic unit will not be sufficient.
    This is mainly since Pod has no self-healing mechanism, does not support scale operations;
    Rolling Update\footnote{\textbf{Rolling Update} \---\ is the process when one updates the Deployment configuration and this update trigger replacements of the Pods with the new desired configuration} or Rollback.
    Deployment has all these features at its disposal.
    Importantly, this controller manages the ReplicaSet, which takes care of self-healing and scale operations.
    This means that the ReplicaSet checks whether the desired state is equal to the current state, such as the number of replicas are equivalent to the current state.
    Additionally, Deployment supplies the remaining properties, which are Rolling Update and Rollback.
    Since Deployment is a fully-fledged object in the Kubernetes API similar to Service, Pod, or Volume, that gives us the ability to define such an object in YAML files, such an object can then be edited and updated, which will trigger a Rolling Update.
    Figure \ref{fig:kubernetes:deploymentReplicaSetPod} shows us the hierarchy of mentioned the controllers.

    \item \textbf{StatefulSet} \---\ The last major controller is StatefulSet.
    This controller has many features in common with Deployment such as reconciliation loop described in~\ref{01:alg:controllerAlg}, scaling operations, and self-healing mechanism.
    The difference between Deployment and StatefulSet are as follows:
    \begin{itemize}
        \item \textbf{storage} \---\ with the Deployment controller, it is possible to specify PersistentVolumeClaim, which is shared between all Pod replicas.
        On the other hand, in the case of StatefulSet controllers, each Pod has its own PersistentVolumeClaim.
        For clarity, one can use Deployment in the case of a stateless application, where each node does not need a unique identity, and in the case of StatefulSet, one can use it in the form of databases (i.e., Cassandra, MySQL) where each node has its own unique storage.
        \item \textbf{unique identity to Pods} \---\ in case of failure remains the same (Deployment will create a new Pod with a completely new name).
        Moreover, StatefulSet guarantees that Pods are created/deleted in order (Deployment does ensure order).
        \item \textbf{scaling operation} \---\  ensures that each new Pod is installed only after the previous one is ready and running.
        This process is repeated until we reach the number of replicas required.
        Figure \ref{02:fig:statefulsetOrderedCreation} illustrates a scaling up scenario, where firstly \emph{Pod\_1} is being deployed and after a while when \emph{Pod\_1} is running and ready, the \emph{Pod\_2} is being deployed.
    \end{itemize}

    \begin{figure}[!h]
        \centering
        \includegraphics[scale=1]{obrazky-figures/02-preliminaries/01-kubernetes/04-statefuset_with_volume}
        \caption{StatefulSet ordered creation of Pods}
        \label{02:fig:statefulsetOrderedCreation}
    \end{figure}

    In Figure \ref{02:fig:statefulsetOrderedCreation}, we see that architecturally StatefulSets has a different self-healing and scaling operations mechanism compared to the Deployment.
    In addition, Volumes plays a significant role in the StatefulSet.
    When the Pod is created, the Statefulset immediately creates an associated volume and attaches this Volume to the Pod.
    This guarantees that the Pod can keep all its information even in the event of an unexpected failure.

\end{enumerate}
    \section{Apache Kafka}
\label{02:sec:title}

This section describes and explains the basics of the Apache Kafka system.
The description is based on two books: \emph{Designing Event-Driven Systems}~\cite{apacheKafkaDesignDistributedSystems} and \emph{Real-Time Data and StreamProcessing at Scale}~\cite{apacheKafkaDefinitiveGuide}.
Moreover, the Kafka streams subsection is based on \emph{Mastering Kafka Streams and ksqlDB Building real-time data systems}~\cite{kafkaStreamsBook}.
We also used Kafka's documentation~\cite{kafkaDocumentation} as the most up-to-date reference.
In these books and documentation, there can be found a more detailed explanation of Kafka itself.

Apache Kafka is an event streaming platform that offers many features like high performance, distribution, commit log service\footnote{\textbf{Commit log} 
---\ is a type of data structure that stores ordered sequences of events.}, and more.
It offers a publish/subscribe system to record streams that are similar to a message queue or enterprise messaging system.
Additionally, it stores record streams in a robust, fault-tolerant way.
Kafka also creates real-time data flows that reliably capture data transferred between systems or applications.
Kafka is widely used by many big companies like LinkedIn, Spotify, Netflix, and Uber.
\subsection{Motivation~\cite{02-bachelor-thesis}}
\label{kafka:motivation}

In the past, companies had applications or systems that share large amounts of data.
Usually, these applications would provide valuable information to another application.
So, there was one source system and one target system.
But what about adding more source and target systems?
Assume an example where one has five source systems and five target systems.
Each source system needs something from each particular target system.
\begin{figure}[ht!]
    \hspace{0.01\textwidth}
    \subfigure[Source and Target systems without Kafka]{\includegraphics[scale=0.60]{obrazky-figures/02-preliminaries/02-kafka/02-kafka-without-depen}}
    \hspace{0.01\textwidth}
    \subfigure[Source and Target systems with Kafka]{\includegraphics[scale=0.60]{obrazky-figures/02-preliminaries/02-kafka/01-kafka-with-depe}}
    \hspace{0.01\textwidth}
    \caption{How to make system more efficient with Kafka}
    \label{fig:withoutAndWithBroker}
\end{figure}

The system without Kafka depicted in Figure~\ref{fig:withoutAndWithBroker}a has twenty-five links, which is not scalable (quadratic complexity).
That is onne of main reasons Kafka was invented.
Lets illustrate the same example with ten systems and Kafka in the middle serving as Middleware\footnote{\textbf{Middleware} \---\ Software, that acts as the middle man between two systems and guarantees interoperability between them.}, which is placed in the middle of these systems.
In that case, each source system only has to bind to the Kafka broker, and all data are delivered by a single link. You can see the updated system in Figure~\ref{fig:withoutAndWithBroker}b.

\subsection{Fundamental concepts}

In this subsection, we describe fundamental concepts of Apache Kafka such as Producer, Consumer, Kafka broker, Kafka cluster, and so on.
The description is based on the \emph{Real-Time Data and StreamProcessing at Scale}~\cite{apacheKafkaDefinitiveGuide} and Kafka documentation~\cite{kafkaDocumentation}.

\begin{enumerate}
    \item \textbf{Kafka broker/cluster} \---\ it is a server application that manages messages that are sent by producers and at the same time obtained by consumers.
    In other words, it takes care of storing the data and the order of the data.
    Sometimes we can see a Kafka broker with names such as Kafka server or Kafka node.
    These names are synonymous with Kafka broker.
    Kafka broker was designed to be horizontally scalable to create a Kafka cluster (two and more Kafka brokers).
    Within a Kafka cluster, there is a single cluster controller.
    The cluster controller takes care of fundamental operations such as assigning partitions to brokers or monitoring for the failure of the Kafka brokers.
    One broker in the Kafka cluster always owns the topic partition.
    This broker is called the leader of this topic partition.
    Of course, this topic partition can be replicated into several Kafka brokers, which will result in its replication and thus data redundancy.
    On the other hand, if the leader Kafka broker fails, the one who has the replicated topic partition will take control and become the new partition leader.
    Figure \ref{02:fig:replicationOfTopicPartition} illustrates this type of scenario, where two Kafka brokers shared data between each other and partitions of the topic are replicated.

    \item \textbf{Producer} \---\ is one of the types of clients that Kafka provides.
    They produce new messages that are sent to a specific topic.
    In general, the client does not need to know to which partition it is necessary to send messages.
    It simply sends messages divided among several partitions.
    Thus, producers represent the entity that creates the data in the Kafka system.
    Kafka also provides the implementation of these clients in several languages such as Java, Go, C++, Python, and many others.
    Kafka also provides a higher level abstractions, which means that it is no longer necessary to create the producers themselves, but those entities are encapsulated in the client.
    These are, for example, Kafka Streams for stream processing or Kafka Connect API for data integration.
    
    \item \textbf{Consumer} \---\ unlike a producer, a consumer or group of consumers tries to consume messages.
    In the consumer configuration, it is necessary to specify the topic from which the consumer will read.
    However, the consumer can also read from a group of topics.
    The consumer maintains an internal offset value that represents a position from where the consumer should read the data from the topic.
    The method that consumers use to read messages is called polling\footnote{\textbf{Pooling} \---\ periodic querying to the server in that case, to the Kafka broker}.
    The consumer group behaves as an single logical unit.
    Kafka does not support reading from one specific partition more than one consumer simultaneously.
    The reason why this concept was created is based on a straightforward question - \emph{How are we able to consumes data concurrently?} Likewise, what is worth mentioning is that we \emph{can not} have more consumers than partitions because, in that type of example, some of them are inactive.
    This concept differs from other messaging solutions and describing why Kafka is so flexible in comparison with the traditional messaging based on AMQP protocol like ActiveMQ or RabbitMQ.
    
    \begin{figure}[!ht]
    \centering
    \includegraphics[scale=0.65]{obrazky-figures/02-preliminaries/02-kafka/05-replication-of-partitions}
    \caption{Kafka topic partition replication scenario in Kafka cluster inspired by \emph{Real-Time Data and StreamProcessing at Scale} \cite{apacheKafkaDefinitiveGuide}}
    \label{02:fig:replicationOfTopicPartition}
    \end{figure}

    \item \textbf{Kafka Topic} \---\ is not a simple concept and includes several parts such as the replication factor, partitions, and more.
    Kafka topic is equivalent to database table as one can see in the Figure~\ref{fig:topicAndDatabaseTable}.
     \begin{figure}[!ht]
    \centering
    \includegraphics[scale=0.80]{obrazky-figures/02-preliminaries/02-kafka/03-database-relation}
    \caption{Equivalence of Kafka topic and database table}
    \label{fig:topicAndDatabaseTable}
    \end{figure}

    Messages are being stored on a specific topic.
    The replication factor is a number, which defines how many replicas will be available on the other brokers from the Kafka cluster.
    Imagine the following scenario \---\ we have a Kafka cluster with three Kafka brokers.
    We create a new topic with a unique name by using an administration client. (In Section~\ref{03:title}, we will talk about alternative ways of creating resources.) The question can be \emph{what happens if we set higher replication factor then we have available Kafka brokers}.
    We are notified that the Topic can not be created because we do not have enough accessible Kafka brokers.
    More about this in~\ref{subSec:strimzi:topicOperator}.
    Partitions are entities that split your Topic into separate parts.
    It means that in each partition, we have different data; using this feature, we allow the consumer to fetch data in a concurrent\footnote{Consumes more than one message at the specific period.} way.
    A partition contains offsets, which serve as ids for the specific messages.
    An offset is an integer value assigned to each consumer indicating the next message, which will be read.
    Consider the scenario when we have one Kafka broker and one Topic with a hundred messages.
    According to offset implementation, the maximum offset value is 100 because it reflects the position of the last message in the Topic.
    If we configure consumers to subscribe to that Topic, it uses the polling method and starts with offset zero.
    The first poll gets twenty messages, so offset move on nineteen and so on. The Figure~\ref{fig:offset} illustrate this scenario.
    In general, we can understand offset as the message index.
    
    \begin{figure}[!ht]
    \centering
    \includegraphics[scale=1.1]{obrazky-figures/02-preliminaries/02-kafka/05-offset-thing}
    \caption{Partition offset}
    \label{fig:offset}
    \end{figure}
\end{enumerate}

\subsection{Kafka Streams}

It is a stream processing tool created by the Kafka community that does expose the low level of the Consumer API and Producer API. These client APIs are very flexible, and the user can create the data processing logic he wants.
However, there is a tradeoff, and it is writing many lines of code.
Unfortunately, we cannot classify these APIs as stream processing APIs because they do not contain primitives that would classify them there, such as \emph{Local} and \emph{Fault-tolerant} state and a set of transformers that work with data (a transformer is an operator that transforms data).

In 2016, Kafka introduced the \emph{Kafka Streams API}, which solved these problems.
Inexperienced users in Kafka Streams would think it is just a matter of sending messages to and from Kafka.
Instead, we can see that Kafka has a part of Producer and Consumer, where it offers a wide range of libraries for data transformation.
Kafka streams also support two crucial operating characteristics:
\begin{enumerate}[itemsep=1mm, parsep=0pt]
    \item \textbf{Scalability} \---\ In Kafka Streams, the smallest unit of work is a single partition.
    If we want to scale the Kafka Streams application, we have to divide Topic into several partitions.
    Practically speaking, you use the Kafka Streams API to deploy multiple instances of an application, each of which will handle a subset of the work.
    For illustration, one Topic has sixteen partitions, and it is up to us how we scale it.
    One scenario could be to deploy two instances, and each of them would trade eight partitions.
    Figure \ref{fig:kafkaStreams} shows example with three partitions.
    
    \begin{figure}[!h]
    \centering
    \includegraphics[scale=0.48]{obrazky-figures/02-preliminaries/02-kafka/07-kafka-streams-with-localstate,thread.pdf}
    \caption{Kafka Streams with local state stores inspired by Kafka Documentation~\cite{kafkaDocumentation}}
    \label{fig:kafkaStreams}
    \end{figure}
    
    
    \item \textbf{Reliability} \---\ If an error occurs on any node, Kafka automatically distributes the load to other nodes.
    However, we must realize that if the node that crashed is the last, we may lose the data if we do not use some Volume or other external storage.
    At the same time, when the node returns the given error is corrected, Kafka will rebalance again.
\end{enumerate}

\begin{figure}[!ht]
    \centering
    \includegraphics[scale=0.78]{obrazky-figures/02-preliminaries/02-kafka/10-kafkaStreamsProces}
    \caption{Micro-batching processing (typical for different systems) inspired by~\cite{kafkaStreamsBook}}
    \label{fig:02-kafkaStreamsProcessingBatch}
\end{figure}

One of the main differences between other similar systems is the processing model that Kafka Streams offers.
These systems, such as \emph{Apache Spark Streaming}\footnote{\textbf{Apache Spark Streaming} \---\ is a extension of Spark API with many transformation methods.} or \emph{Trident}\footnote{\textbf{Trident} \---\ high-level abstraction for stream processing based on the Apache Storm. It provides multiple transformation methods such as filters, grouping, and aggregations.}, use micro-batching, which occurs very much in machine learning where work is divided into several batches.
These groups are then loaded into memory then emitted at a pre-selected interval (typically $1s$ or less).
Figure~\ref{fig:02-kafkaStreamsProcessingBatch} shows a micro-batching strategy, where one can see that events are coupled into groups.
By contrast, Kafka Streams offers us event-at-a-time processing, where events are processed as soon as they arrive.
This approach gives us low latency and is considered true data streaming. Figure~\ref{fig:02-kafkaStreamsProcessingEvent} illustrates the event-at-a-time processing strategy.
\begin{figure}[!ht]
    \centering
    \includegraphics[scale=0.88]{obrazky-figures/02-preliminaries/02-kafka/11-kafkaStreamsProces2}
    \caption{Kafka Streams uses event-at-a-time processing inspired by \emph{Mastering Kafka Streams and ksqlDB Building real-time data systems}~\cite{kafkaStreamsBook}}
    \label{fig:02-kafkaStreamsProcessingEvent}
\end{figure}

Kafka Streams is thus a set of libraries that offer developers incredible power over data processing.
Additionally, it has a model of parallelism, where the smallest logical unit is partition.
Easily scalable by either increasing or decreasing partitions, and lastly, Fault tolerance is rooted in Kafka itself (dependent on Topic replicas).
This collection of characteristics make it the perfect choice for today's data intensive applications.
These types of applications could be, for instance:
\begin{itemize}[itemsep=1mm, parsep=0pt]
    \item email tracking, monitoring,
    \item chat infrastructure (Slack), virtual assistants, chatbots,
    \item machine-learning pipelines (Twitter),
    \item smart home (IoT sensors).
\end{itemize}
There are many such types of applications.
However, what brings together all the examples is real-time data processing.

\subsection{Kafka Connect}

One of the most critical questions that every data engineer has is: "\emph{How to move data from Kafka to a datastore or vice versa?}".
Moreover how to create data pipelines that connect several systems, for instance, by selecting data from Twitter and then sending it to Elasticsearch or other external storage.
Of course, Kafka will play a middleware role in this data transfer.
We can answer the previous question and solve the data integration problem thanks to the \emph{Kafka Connect} component.

Kafka Connect offers a large number of features that are transparent to the users.
These include configuration, parallelisation, error handling, and much more.
Moreover, for data integration, Kafka Connect offers two types of connectors.
Connectors are already predefined templates.
These connectors need metadata information to work.
We give this connector information such as the names of one or more Topics to follow.
In addition, these are attributes such as the connector class, number of tasks executed in parallel, and the connector URL.
The first such type of connector is Kafka Connect Source, which obtains the data from the datastore.
Information about what datastore and other metadata are provided in the connector configuration files.
In case the data in the datastore are changed, the data is automatically sent to one or more Topics.
The second type is the Kafka Connect Sink, which is analogous to the Source connector.
In the connector configuration, we define which datastore it should add data to and from which Topic it should monitor changes.
When Topic changes his state, this data is automatically pushed into the given datastore.
The simplest examples of connectors already mentioned above are the \emph{FileSource} and \emph{FileSink} connectors.

However, to properly understand Kafka Connect, it is necessary to know how the following fundamental mechanisms work:

\begin{figure}[!ht]
    \centering
    \includegraphics[scale=0.8]{obrazky-figures/02-preliminaries/02-kafka/12-all-in-one}
    \caption{The entire Apache Kafka ecosystem.}
    \label{fig:02-ecosystem-of-kafka}
\end{figure}

\begin{enumerate}
    \item \textbf{Connector} \---\ As mentioned above, the connectors are used to transfer data to and from Kafka.
    Among the essential responsibilities of connecting connectors to a given datastore, it maps the data structure that the external storage has at its disposal and decides how many tasks (threads) will run simultaneously during the transformation.
    \item \textbf{Worker} \---\ This entity is responsible for the REST API available to Kafka Connect.
    They check REST API requests and respond accordingly.
    If a worker error occurs in any way, the other workers in Kafka Connect will know this information as soon as possible and then perform rebalance and redistribute the work.
    \item \textbf{Data model and converters} \---\ Kafka Connect API contains endpoints of data objects and the scheme.
    These objects can be database tables, JSON, XML, AVRO schemas.
    Converters transform this schema to a Connect Schema object.
    Subsequently, this Connect Schema object is sent to the target system.
    There are currently many such converters available.
\end{enumerate}

All the mentioned Kafka components can be divided into three stages.
The first milestone was the emergence of a new messaging system with basic functionality and no enterprise libraries.
These included components such as Kafka Broker, Topic, Consumer, and Producer.
The lack of libraries and writing vast amounts of code in data processing brought Kafka Streams.
Kafka Connect solved data integration problems between other systems.
Finally, the Kafka Mirror Maker 2 concept came along, which improved the Kafka Mirror Maker predecessor with many capabilities.
It was a way to move data from one Kafka cluster to another.
The whole Kafka ecosystem is not trivial.
Figure \ref{fig:02-ecosystem-of-kafka} shows these stages starting with the Kafka Broker, Producer, Consumer, and Topic.
There are many other parts, such as Kafka Quotas or Kafka Rebalance features.
Nonetheless, in the thesis, we do not deal with Rebalance, Mirror Maker, or Kafka Quotas, and therefore it is not necessary to explain them in detail.
However, in case of interest, I recommend the previously mentioned literature.
    \section{Strimzi}
\label{03:title}

This section describes the fundamental parts of the Strimzi project. Moreover, it explains the whole architecture with all Operators (i.e., Topic, User, Cluster). The description is based mainly on Strimzi documentation and blog posts\cite{strimziDoc, strimziBlogPosts}.

The information described in Sections \ref{01:sec:title}, \ref{02:sec:title} was a precursor to a complete understanding of the Strimzi system. Strimzi is an Apache Kafka orchestrator in the Kubernetes environment. It is therefore a collection of operators that simplify work with Kafka. The Operator in Kubernetes is a component that is always in one of the following three states:
\begin{itemize}[itemsep=1mm, parsep=0pt]
    \item \textbf{Observe} \---\ gain the desired and current state,
    \item \textbf{Analysis} \---\ compares these two states and finds the differences,
    \item \textbf{Act} \---\ subsequently, if the given differences were found, it will do a reconciliation that will make the current and desired state identical
\end{itemize}

One can understand these Operators as a superset of the Deployment controller, which, like other controllers, followed this \ref{01:alg:controllerAlg} algorithm. The main difference is that the Operator oversees \emph{Custom Resources - CR}. Custom Resource is an extension of the Kubernetes API. These CRs define your application objects in the Kubernetes environment. Moreover, this is associated with the \emph{Custom Resource Definition}, which declares what values and types a given Custom Resource can acquire. We can also imagine that Custom Resource Definition is a template comparable to classes in the Object-Oriented programming world and Custom Resource is an instance of the class. Strimzi has defined a Custom Resource Definition for each Kafka component we described in section \ref{02:sec:title} except for clients. For example, for the KafkaBroker component, Strimzi has its Custom Resource Definition and others. 

\begin{figure}[!h]
    \hspace{0.01\textwidth}
    \subfigure[Example of Kafka Custom Resource Definition (Unnecessary parts omitted for brevity).]
        {\includegraphics[scale=0.8]{obrazky-figures/02-preliminaries/03-strimzi/01-kafka-crd.pdf}}
        \label{03:fig:kafkacrds}
    \hspace{0.01\textwidth}
    \subfigure[Example of Kafka Custom Resource]
        {\includegraphics[scale=0.8]{obrazky-figures/02-preliminaries/03-strimzi/02-kafka-cr.pdf}}
        \label{03:fig:kafkacr}
    \hspace{0.01\textwidth}
    \caption{Kafka Custom Resource Definition and Kafka Custom Resource (Class and Instance)}
    \label{03:strimzi:fig:crds}
\end{figure}

Figure \ref{03:strimzi:fig:crds} illustrates the mentioned Custom Resource and Custom Resource Definitions. In Figure \ref{03:strimzi:fig:crds} (left side) one can see Kafka Custom Resource Definition that shows several essential parts:
\begin{itemize}[itemsep=1mm, parsep=0pt]
    \item \textbf{labels.app.strimzi}  \---\ every Kafka Custom Resource in Kubernetes contains this label, and with that, it is easier to find these resources
    \item \textbf{spec.names.kind.Kafka} \---\ by this attribute, we specify how the Custom Resource type will be uniquely named. In this case, the label is Kafka.
    \item \textbf{spec.scope.Namespaced} \---\ type of environment scope. It distinguishes between Custom Resource, which works multi-namespace or single-namespace. Because Kafka Custom resource has value Namespaced (single-namespace), it can work in one namespace. On the other hand, we also know the Custom Resource, which can have the scope set to Cluster (multi-namespace), which means that they will observe all the namespaces that the Kubernetes cluster has.
    \item \textbf{spec.schema} \---\ this is the whole declaration of the Custom Resource Definition. In the child nodes, we can see what types the individual attributes must comply with and the restrictions on the given types. For example, the attribute \emph{replicas} we have a restriction that it must have at least one replica and similarly for other attributes. For imagination, the current version of Strimzi 0.25.0 Kafka Custom Resource Definition consists of 7000 lines of code.
\end{itemize}

On the other hand, we have Kafka Custom Resource (Figure \ref{03:strimzi:fig:crds} - right side), which includes parts worth mentioning:
\begin{itemize}[itemsep=1mm, parsep=0pt]
  \item \textbf{apiVersion} \---\ This is the REST API offered by the Custom Resource Definition. The prefix must also match the value found in Kafka Custom Resource Definition in spec: group. 
  \item \textbf{metadata.name} \---\ Custom Resource name,
  \item \textbf{spec.kafka.version} \---\ version of Kafka to be used,
  \item \textbf{spec.kafka.replicas} \---\ number of Kafka Pods to be deployed,
  \item \textbf{spec.kafka.listeners} \---\ types of listeners to be supported by a given Kafka instance. In this case, we see two types, one with plain communication listening on port 9092, and the second listener with encrypted communication using TLS technology and listening on port 9093.
  \item  \textbf{spec.kafka.config} \---\ these are additional configuration features that are added to Kafka,
  \item  \textbf{spec.kafka.storage} \cite{strimziStorageBlogPost} \---\ the type of used storage. Kubernetes has been supporting two storage types for a long time. In our case, it is ephemeral storage. Ephemeral storage is usually a directory somewhere in the operating system on our Kubernetes node. It works the same as the temporal directory. There are also risks associated with this, when the Kubernetes node may crash, and the data stored in the given storage will be lost. The same thing will happen if we get a running Pod that will use ephemeral storage. In case of a restart, together with the new Pod, empty storage will be created, not containing the previous data. The second type of storage is Persistent, which eliminates these risks.
  \item \textbf{spec.zookeeper.replicas} \---\ number of Zookeeper Pods to be deployed,
  \item \textbf{spec.entityOperator} \---\ configuration for Entity Operator.
\end{itemize}

\subsection{Architecture}

Strimzi architecture consists of two larger units, where the first unit is Kafka architecture and other components with which it communicates. The second unit is the Operators architecture, consisting of a Cluster Operator, an Entity Operator, a Topic Operator, and a User Operator. These Operators then have control loops, which control the already defined Custom Resources. (i.e, Kafka User, Kafka Topic, Kafka and Kafka Connect, Kafka Bridge, Kafka Mirror Maker, Kafka Mirror Maker 2, Kafka Rebalance)

Kafka Architecture consists of several components, each of which performs specific tasks. Zookeeper is one of the most significant dependencies for Kafka and limits it in several areas. Whether it is scalability, metadata management, or deployment itself, the answer to these problems was also in 2020 Kafka Improvement Proposal abbreviated KIP\footnote{KIP-500 \---\ removal of Zookeeper with replacing him with self-managed metadata quorum \url{https://cwiki.apache.org/confluence/display/KAFKA/KIP-500}}, which was accepted. Thus, Kafka 3.0 should already be without Zookeeper's dependency. Its responsibilities include, for example, leader election of partitions or storing the status of Kafka Brokers or Consumer offsets. Clients in Figure \ref{03:fig:strimziKafkaArchitecture} are classically Producer and Consumer as we know from the section \ref{02:sec:title}, so their objective is clear. On the other hand, HTTP clients communicate with Kafka Bridge and thus connect the Kafka cluster and the clients themselves. 
\begin{figure}[!ht]
    \centering
    \includegraphics[scale=0.75]{obrazky-figures/02-preliminaries/03-strimzi/03-strimzi-kafka-architecture (1).pdf}
    \caption{Strimzi Kafka architecture}
    \label{03:fig:strimziKafkaArchitecture}
\end{figure}
It communicates by default via the REST API, and the user can create, delete, update Consumer, Producer, Topic and similar resources that Kafka Bridge offers.  So Kafka Bridge is nothing more than an HTTP proxy that integrates HTTP clients with a Kafka cluster. Another part of the Kafka architecture is the Kafka Exporter and is used to extract data from the Prometheus\footnote{Prometheus \---\ open-soured metrics-based project. Moreover, it provides an alerting system with incredible features, in case of interest \url{https://prometheus.io/}}. Then we have Kafka Connect and Kafka Mirror Maker, where we described the meaning of these components in the section \ref{02:sec:title}. The last essential component, especially for the overall balancing of the Kafka cluster, is Cruise Control. This component collects data on CPU pull-out, partitions status, and many other metrics. Cruise Control creates an information model and analyzes when necessary to perform balancing and rearrange the load. Everything we have described shows the following Figure \ref{03:fig:strimziKafkaArchitecture}.

The second part in the Strimzi architecture is the collection of Operators. In the beginning, we described what such an Operator does (reconciliation/control loop). Strimzi contains three Operators, where hierarchically the highest is Cluster Operator, which manages Kafka, Kafka Mirror Maker, Kafka Mirror Maker 2, Kafka Connect, Kafka Rebalance and Kafka Bridge Custom Resources. Furthermore, since Kafka Custom Resource encapsulates the Entity Operator (Topic and User Operator running in the same Pod but different containers) and Zookeeper, the Operators mentioned above are also deployed with each Kafka Custom Resource deployment. Figure \ref{03:fig:strimziOperatorsArchitecture} illustrates whole Strimzi ecosystem, for which is Cluster Operator responsible.

\begin{figure}[!ht]
    \centering
    \includegraphics[scale=0.60]{obrazky-figures/02-preliminaries/03-strimzi/04-stirmziOperatorsArch.pdf}
    \caption{Strimzi Operators architecture with Strimzi ecosystem}
    \label{03:fig:strimziOperatorsArchitecture}
\end{figure}

Topic Operator takes care of creating, deleting, updating individual Topics. It is also necessary to mention that Topic Operator ensures synchronization between Custom Resource Topic and Topic (located inside Kafka container) and keep them in sync. \emph{For
instance, assume the scenario where the user changes different topic properties in Kubernetes
but simultaneously in Kafka itself. Also, imagine another scenario where one change topic property at the same time. The first action is considered as allowed, and the solution for this is a 3-way diff (more about this method in section 2.19). In general, this method constructs these two differences' union and finds out where the intersection is not empty. The second one is treated as incompatible change. It must deterministically select by some winner policy implemented inside Topic Operator}.

The User Operator is responsible for the Kafka User Resource, which specifies authentication and authorization for individual components. It can be, for example, the Producer that can not change data in a Topic with a particular name or prefix name. In other words, we can define read, read and write rules for Topics. In addition, we can create different types of Kafka Users, which support authentication such as TLS or SCRAM-SHA. Nevertheless, if we use SCRAM-SHA authentication, we must also configure one of the Kafka Broker listeners. Noteworthy, when one creates Kafka Custom Resource, then immediately User Operator creates associated Secret with the credentials. These credentials are then submitted to the Consumer or Producer configuration. Credentials ensure that the Producer or Consumer can connect to Kafka Broker and send or receive messages. In authorization, several components can also be used, such as ACLs (access control lists). For more complex rules, there is support for the Keycloak or Hydra authorization server. Another exciting feature is User quotas, ensuring that one client will never control the entire Kafka Broker, and the total load will be limited.

%\subsection{Deep inside Strimzi TODO: decide if this is needed...}

%ROLLLING UPDATE a ROLLBACK mechanizmus zmenit a popisat ako to funguje. Co to je referencovat z Kuberernetes. Kedy sa vykonava v pripade Strimzi. Algorihmus jebnut a hotovo. To iste s Rollbackom.
    \section{Strimzi system tests}
\label{02:sec:strimzisystemtests}

This section describes the basics of the Strimzi system tests.
At first, we will go through a short description of how we test Strimzi.
Then in Section~\ref{02:subsec:strimziJunit5relation:execution} we explain the fundamentals of JUnit5 and how tests are discovered and executed.
Lastly, in Section~\ref{02:subsec:strimzisystemtestsexecution} we explain Strimzi system test management and execution flow.

Overall testing begins, as we know from textbooks with unit testing.
Subsequently, if this phase is successful, we move on to integration tests and then to system tests.
Of course, the most time-consuming is the system tests, which in our case take about 40 hours to complete.
\begin{figure}[!ht]
    \centering
    \includegraphics[scale=0.70]{obrazky-figures/02-preliminaries/04-strimzi-system-tests/01-architecture-overall}
    \caption{Strimzi system tests top-level component architecture}
    \label{02d:fig:strimzisystemtestarch}
\end{figure}
The testing phases are dependent on each other in the order in which they are executed.
For instance, integration tests will not run if unit tests fail, similarly to integration and system tests.
Furthermore, system tests run on multiple infrastructures such as \emph{Openstack}, \emph{Microsoft Azure} or \emph{Amazon Web Services}.
There are certain limitations to the set of tests on each of these infrastructures.
Since these are Kubernetes system tests, it is essential to realise that the total load on the computing resource is enormous.
At the same time, the preparation of resources and their cleaning is time-consuming.
Therefore, our system tests have two essential parts.
The first is resource classes that provide the user interface for creating, retrieving, deleting, and updating these resources.
Moreover, we have three independent stacks that serve as resource storage.
These stacks are responsible for storing all resources based on the test case.
Furthermore, the deletion of these resources is transparent for the user just as if it is a resource created in a \emph{@BeforeAll}\footnote{\textbf{@BeforeAll } \---\ is JUnit5 annotation, where one specifies what must be executed before all tests in the test suite.} annotated method.
The second fundamental part is auxiliary classes such as Utils\footnote {\textbf{Utils} \---\ type of class that consists of static methods, which in general dynamically wait for a specific event. For instance, waiting for Rolling Update, if one changes Kafka configuration}, Apache Kafka clients for external communication, Kubernetes client offering an API for communication with the Kubernetes cluster and finally classes such as \emph{Constants} and \emph{Environment}. This can be seen in Figure~\ref{02d:fig:strimzisystemtestarch}.

\subsection{JUnit5 relation and execution of test cases}
\label{02:subsec:strimziJunit5relation:execution}

Junit5 Engine handles the entire implementation and management of the test lifecycle.
The Engine facilitates the discovery and execution of tests for a specific programming model.
In other words, it is the entity in charge of discovering and executing tests.
Discovering can be thought of as scanning all the classes and methods in specific directories.
The Engine has specified in advance which signatures to include in the test tree.
In the case of the Junit5 Engine, it is a sequence of chaining methods, which gradually add all classes (test suites) and methods (test cases) to the test tree.
They also add the test types defined by them (i.e., \emph{@TestFactory}, \emph{@ParametrizedTest}, \emph{@TestTemplate}).
Everything is depicted in Algorithm~\ref{02:alg:selectorresolver}.

\begin{algorithm}[H]
    \label{02:alg:selectorresolver}
    \caption{Junit5 Engine: Discovery selector resolver}
    \begin{algorithmic}[1]
        \Procedure{resolveSelectors}{DiscoveryRequest request, Descriptor descriptor}
            \State {EngineDiscoveryRequestResolver.$<$JupiterEngineDescriptor$>$builder()}
            \State {.addClassContainerSelectorResolver(new IsTestClassWithTests())}
            \State {.addSelectorResolver(c $\rightarrow$ new ClassSelectorResolver(classFilter, config))}
            \State {.addSelectorResolver(c $\rightarrow$ new MethodSelectorResolver(config))}
            \State {.addTestDescriptorVisitor(c $\rightarrow$ new ClassOrderingVisitor(config))}
            \State {.addTestDescriptorVisitor(c $\rightarrow$ new MethodOrderingVisitor(config))}
            \State {.addTestDescriptorVisitor(c $\rightarrow$ TestDescriptor::prune)}
            \State {.build();}
            \State {.resolve(request, descriptor);}
        \EndProcedure
    \end{algorithmic}
\end{algorithm}

Once the resolver is created, we can run the following algorithm, using the resolver and creating the already mentioned tree of \emph{TestDescriptors}.
Here is a detailed description of how the algorithm works:

\begin{enumerate}[itemsep=1mm, parsep=0pt]
    \item Enqueue all selectors in the supplied request to be resolved.
    \item While there are selectors to be resolved, get the next one.
    Otherwise, the resolution is finished.
    \begin{enumerate}
        \item iterates over recorded resolvers in the directive that they were recorded in and discover the foremost one that yields a resolution other than unresolved().
        \item If such a resolution exists, enqueue its selectors.
        \item For each exact match in the resolution, expand its children and enqueue them as well.
    \end{enumerate}
    \item Iterate over all registered visitors and let the engine test descriptor accept them.
\end{enumerate}

The second phase after the correct scan of test cases that the user wants to perform is execution.
In this case, TestEngine already has a TestDescriptor in which all the information needed to run is available.
At this stage, the TestEngine must always notify the Junit5 platform of the success or failure of the test case.
Moreover, Engine instantiates the \emph{SameThreadHierarchicalTestExecutorService} class, which ensures that each test is performed sequentially.

\subsection{Strimzi system test management and execution flow}
\label{02:subsec:strimzisystemtestsexecution}

In the previous Section~\ref{02:subsec:strimziJunit5relation:execution}, we described the intricate parts of loading and the type of tests performed.
In the case of the Strimzi part, adding several mechanisms (i.e., creation of Kubernetes cluster, communication with Kubernetes cluster, management of Kubernetes resources, waiting for conditions) is necessary.
We solve all these parts in Strimzi.
We create Kubernetes clusters in several ways as we test the product on several infrastructures.
For example, on Microsoft Azure, we create a Minikube (a subset of the Kubernetes cluster, one-node cluster) with approximately eight CPUs and 16GB of RAM. In Openstack, we typically create a six-node cluster with three master nodes and three worker nodes.
Each has eight CPUs and 16GB available (similarly to Amazon Web Services).

Communication with the Kubernetes cluster is guaranteed by the Fabri8 Kubernetes client \url {https://github.com/fabric8io/kubernetes-client}.
This client provides a Java client with many methods that communicate directly via the Kubernetes REST API.
Most methods are designed to create, update, delete and retrieve a given resource.
In practice, we will also encounter the term CRUD methods.
To illustrate, we can imagine getting all the namespaces on a given Kubernetes cluster.
All namespaces are obtained using the command \emph{client.namespaces().List();}.

The overall orchestration of Kubernetes resources is handled by the \emph{ResourceManager} class and its additional resource classes.
As we wrote at the beginning of Section~\ref{02:sec:strimzisystemtests}, it includes three stacks where the main/pointer stack points to the method or class stack based on context.
For example, suppose the execution is located in \emph{@BeforeAll} or \emph{@AfterAll} annotation, we add elements to the class stack.
In other scenarios, such as in the test case or \emph{@BeforeEach}, we add elements to the method stack.
This data structure will guarantee the correct order of resources deletion at the end of each test or test class.
This is because we want to delete resources in the order they were created.
So if we create first Kafka, Producer, and lastly Consumer, then in the clean-up phase, we will first delete Consumer, Producer, and finally Kafka.
Thus, the user who creates the test cases does not have to delete individual resources created for the entire test.
In other words, the clean-up phase is transparent to the user.
However, if one wants to delete the resource explicitly, it is possible via the following command \emph{ResourceType.delete(name)}.
Algorithm~\ref{02:alg:deletionalg} defines clean-up phase.

\begin{algorithm}[H]
    \label{02:alg:deletionalg}
    \caption{ResourceManager generic deletion algorithm}
    \begin{algorithmic}[1]
        \Procedure{deleteLater}{MixedOperation$<$T, ?, ?, ?$>$ operation, T resource}
            \State switch(resource.getKind()) \{
            \State \hspace{2em} case Kafka.RESOURCE\_KIND$:$
            \State \hspace{4em} pointerResources.push(() $\rightarrow$ {
                \State \hspace{4em} operation.inNamespace(resource.getMetadata().getNamespace())
                \State \hspace{4em} .withName(resource.getMetadata().getName())
                \State \hspace{4em} .withPropagationPolicy(DeletionPropagation.FOREGROUND)
                \State \hspace{4em} .delete();
                \State \hspace{4em} waitForDeletion((Kafka) resource);
                \State \hspace{2em} });
            \State \hspace{2em} break;
            \State \hspace{2em} case KafkaConnect.RESOURCE\_KIND$:$
            \State \hspace{2em} case KafkaMirrorMaker.RESOURCE\_KIND$:$
            \State \hspace{2em} \dots (other resource)
            \State \hspace{4em} // similar to Kafka resource
            \State \hspace{2em} default:
            \State \hspace{4em}        pointerResources.push(() $\rightarrow$  {
                \State \hspace{4em}           operation.inNamespace(resource.getMetadata().getNamespace())
                \State \hspace{4em}                  .withName(resource.getMetadata().getName())
                \State \hspace{4em}                  .withPropagationPolicy(DeletionPropagation.FOREGROUND)
                \State \hspace{4em}                  .delete();
                \State \hspace{2em}     });
            \State \}
            \State return resource;
        \EndProcedure
    \end{algorithmic}
\end{algorithm}


By contrast, when creating any resources, the user has at his disposal, for example, KafkaResource, KafkaTopicResource, and the like.
Each of these classes contains predefined templates that include specific configuration settings.
A typical example is Kafka, which can be seen in Listing~\ref{kafka:resource}.

\begin{lstlisting}[language=Java,label=kafka:resource,caption=Default Kafka Custom Resource in KafkaResource\.class,frame=tb]
private static KafkaBuilder defaultKafka(Kafka kafka,
            String name, int kafkaReplicas, int zookeeperReplicas) {
    return new KafkaBuilder(kafka)
        .withNewMetadata()
            .withName(name)
            .withNamespace(ResourceManager.kubeClient().getNamespace())
        .endMetadata()
        .editSpec()
            .editKafka()
                .withVersion(Environment.ST_KAFKA_VERSION)
                .withReplicas(kafkaReplicas)
            .endKafka()
            .editZookeeper()
                .withReplicas(zookeeperReplicas)
            .endZookeeper()
        .endSpec();
}
\end{lstlisting}

Another part of the Strimzi system tests is the wait for methods mechanism.
It is used primarily in scenarios where it is necessary to wait for an event to occur.
An example could be waiting for a Rolling Update to occur when Kafka's original Statefulset changes. \begin{figure}[!ht]
    \centering
    \includegraphics[scale=0.70]{obrazky-figures/02-preliminaries/04-strimzi-system-tests/02d-strimzisystemtest-sequence-execution}
    \caption{Strimzi system tests execution flow}
    \label{02d:fig:strimzisysmtetest:execution}
\end{figure}
The second example could be while waiting for a particular pessimistic scenario (i.e., the Cluster Operator Pod will switch to the CrashLoopBack state, and the KafkaBridge Deployment Status will contain the text in the message).

So if we summarise everything we have learned.
It all starts with scanning the test directory, which provides a tree of TestDescriptors.
This is the primary responsibility of TestEngine, which uses selectors to filter out all test cases and the visitors who accept the individual test cases.
As soon as we have a tree available consisting of TestDescriptor nodes, TestEngine starts execution.
This execution is sequential for each test case.
At the same time, thanks to our management and defined resources, we can communicate with the Kubernetes cluster.
For example, in Figure~\ref{02d:fig:strimzisysmtetest:execution} we can execute n Test cases where Test Suite 2 is currently executed and specifically Test Case 1.
The attentive reader will realise that the execution model is sequential due to Java's main thread, the primary thread identifier.
    \chapter{Theory of paralelization}

This chapter describes the fundamental theory of parallelisation (i.e., Amdahl's law (\ref{04:amdalhlaw}), Shared memory (\ref{04:sharedmemory}), Threads and Processes (\ref{04:processesandthreads}), Mutual Exclusion (\ref{dependenciesandprotection}), Synchronization (\ref{04:synchronization}), Asynchronous tasks (\ref{04:asynctaks})). This chapter is based on the following books \emph{An Introduction to Parallel Programming} \cite{introductionToParallelProgramming} and \emph{The Art of Multiprocessor Programming} \cite{artOfMultiprocessorProgramming}.

In the past, computers did not have an operating system. They could only execute one program at a time from start to end. The programmers of the time were as respected as the virtuoso in music and the arts. Writing such programs has been highly challenging. This problem was solved by developing operating systems that can run several processes (programs). 
Processes use the so-called variant of coarse-grained communication. 
Coarse-grained communication includes primitives such as sockets, signals, semaphores, shared memory, and files. This variant allows them to communicate with each other using signals, files or shared memory.  These processes were virtually von Neumann computers, which contained their own memory space that included instructions and data. Subsequently, the processes executed these instructions according to the semantics of the assembly language. The last part was a set of I/O operations to communicate with each other. If we connect all the parts, we will have a model called Sequential. This model is used by most of today's programming languages.  Specifically, the sequential programming model is intuitive because it creates a sequence of operations that follow each other, thus making the expected result. However, this model has its limitations on performance and time consumption on specific tasks. During the twentieth century, technological advances brought a regular increase in the clock's speed, so that the software really "accelerated" itself over time. However, this scenario is not repeated in the twenty-first century. Today's advances in technology bring about a regular increase in parallelism, but only a slight increase in clock speeds. The use of this parallelism is one of the great challenges of modern informatics.

\section{Amdahl's law}
\label{04:amdalhlaw}

If we imagine ourselves as a team that would like to migrate from a single-processor program to a multi-processor program, it would be perfect to be sure that if we embark on the parallelisation of such a system, it will pay off. Moreover, many people live in a bubble, where they think that if we build a multi-processor program from a one-processor program and run it on 3-cores, the overall acceleration will be three times. This is an illusion, and we will never get such a result. The main problem is due to the division of labour which is not uniform for all parts. For clarity, we will illustrate with an example. Imagine that one has to construct a home table. In this case, it is a sequential approach. However, adding four identical tables (so there will be five) will take five times more time for one. Suppose four friends come to help him (we assume they are just as skilled and start simultaneously). The acceleration for such identical tables will be five times. Nevertheless, everything gets complicated if the tables are not the same. For example, the second table will be more complicated to build and take more time than the others. Furthermore, the first will be smaller, and thus the total time will be lower. This implies that the acceleration will not be close to 5-times, but it will probably be almost 3-times. This kind of analysis is crucial for concurrent computation, and thanks to Mr Amdahl, we have a formula for such calculation. It is called Amdahl's law, which can be seen in Equation \eqref{eqn:einstein}.
\begin{equation}
    \label{eqn:einstein}
    S = \frac{1}{1 - p + \frac{p}{n}}
    \tag{1}
\end{equation}
The formula defines the acceleration \emph{S}, which depends on the quantities \emph{n} and \emph{p}. \emph{n} is a non-zero positive number that represents the number of concurrent processors performing the same job. \emph{p} is a non-zero positive number that defines how much work is done in parallel. The sequential part that cannot be parallelised is defined as the difference between the total work and the work that can be parallelised (\emph{1 - p)}. The parallel component is expressed as the ratio of the parallel part and the number of competitors by the processor (\emph{p / n}). So if we sum up these two parts, we get the total time performed by parallel computation (\emph{1 - p + p / n}). Finally, we have to put the ratio between the sequential (single-processor) time and the parallel time, and we get already mentioned Equation \ref{eqn:einstein}. If we apply this formula to the previous example with five friends who want to build five tables, we get such an Equation \eqref{eqn:amdalhinpractice}.

\begin{equation}
    \label{eqn:amdalhinpractice}
    S = \frac{1}{1 - \frac{3}{5} + \frac{\frac{3}{5}}{\frac{5}{1}}} = 25/13 =\sim 2x \; acceleration
    \tag{2}
\end{equation}

Before we dive into the overall terminology and discuss the Critical section, Mutual exclusion, etc., it is necessary to know what the program is correct. The correctness of the program consists of two essential properties. The first is the safety property, which states: "\emph{Bad thing never happens}". To illustrate, imagine the concurrent program never end up in a deadlock\footnote{\textbf{Deadlock} \---\ is a situation where two processes or threads enters a waiting state because a requested system resource is held by another waiting process, which in turn is waiting for another resource held by another waiting process (toto prepisat)}. The second is the liveness property, which tells us: "\emph{An excellent thing will happen eventually}". For instance, the program always terminates. Thus, if we combine these two properties, then we say that the program is correct.

\section{Shared memory}
\label{04:sharedmemory}

The first aspect is memory. One needs to understand how memory is organised and how a computer accesses individual data. The speed of memory in a computer is usually much slower than the speed at which the processor operates, and if one processor overwrites data in memory, the others must wait. In this type of memory, all processors access the same memory in the global address space. 
\begin{definition}
  \textbf{Shared memory} \---\ is a type of memory, where all CPUs has access to the same address space.
\end{definition}
So if one processor makes a change to the data, all the other processors will know about it. The shared memory architecture is classified as UMA (Uniform memory access) and NUMA (Non-uniform memory access). This classification tells us how the individual processors are connected to the memory and how fast the data can be accessed. The wise reader might realise that memory access will be the same for all processors in Uniform memory access. While at Non-uniform memory access, the time will be different. In the UMA architecture, each processor has its cache memory, storing the most frequent data. However, if the processor uses cache memory, there is a very high risk for cache coherence\footnote {\textbf{Cache coherence} \---\ this is a situation where one of the processors obtains a value from shared memory and makes a change in its cache memory and fails to do so. Update to shared memory (while the other processor reads a value that has not yet been updated and will therefore work with the wrong value)}. Fortunately, this cache coherence is handled by hardware in multicore processors.

\section{Processes and Threads}
\label{04:processesandthreads}

If one imagines a shell script with a predefined set of instructions (bash commands), the moment someone runs it, it becomes a Process running in the Operating System. 
\begin{definition}
  \textbf{Process} \---\ is a dynamic object, which has its own global address space. 
\end{definition}
We can also imagine it as a static entity (written shell script) and a dynamic entity (shell script execution). In general, the Process contains program code, its data, and status information. Each Process is independent of the other and has its own address space in memory. On the other hand, there is also a subset of the Process, and it is a thread. 
\begin{definition}
  \textbf{Thread} \---\ is a lightweight variant of the Process that has an independent execution path and shares code and data within a specific Process. 
\end{definition}
Each thread must be part of a process. Thus, the data we work with is shared with all threads inside the Process. Furthermore, each thread has an independent path of program execution. One can imagine a thread as a lightweight variant of the Process. It is well known that threads take up less memory. Moreover, the operating system can switch faster between individual threads than between processes (context switching\footnote {\textbf{Context switching} \---\ it is a situation where the Process scheduler finds out that some processes have spent a fair share of its time on the processor and swap it with the different Process. When this happens, the Operating system stores the state of Process or thread and then load the state of a different process.}). In general, threads can be in one of four states:
\begin{enumerate}[itemsep=1mm, parsep=0pt]
        \item \textbf{New} \---\ If the main thread spawns a new thread, that thread will be in the \emph{New} state. Moreover, the descendants of the main thread can further create a tree hierarchy of new threads.
        \item \textbf{Runnable} \---\ If one creates a thread, it automatically acquires the \emph{New} state. Subsequently, in order to change to the \emph{Runnable} state, it is necessary to run the thread explicitly.
        \item \textbf {Blocked} \---\ If a thread needs to wait for an event, it switches to the \emph{Blocked} state. This is very useful in terms of resource utilisation. If the event occurs, the operating system assigns the CPU time and returns the thread to the \emph{Runnable} state.
         \item \textbf{Terminate} \---\ The thread returns to the \emph {Terminate} state if it was previously aborted abnormally (i.e., using inter-process communication) or complete its execution.
\end{enumerate}

\section{Dependencies and Protection}
\label{dependenciesandprotection}

One of the main challenges in parallel programming is detecting dependencies between threads. Imagine a situation where two threads access the shared variable \emph{x}. \emph{Thread A} reads a value from the shared variable \emph{x} and starts execution. Subsequently, the scheduler switches the context, and \emph{Thread B} reads the value of the shared variable \emph{x}. Then \emph{Thread B} modifies the value of \emph{x = 10}. The scheduler switches the context again, and \emph{Thread A} is currently operating with the wrong value. This is one of the possible faults that can occur in parallel programming. With this example, we have described the Data race failure.

\begin{definition}
    \textbf{Data race} \---\ is a situation where two or more concurrent threads access the same address space, and one of these threads are changing it.
\end{definition}
Fortunately, as programmers, we can eliminate such errors. Process begins with the detection of critical sections in the code.
\begin{definition}
	\label{04:criticalsection}
 	\textbf{Critical section} \---\ section of code, where two or more concurrent threads have write-access (simultaneously) and at least one of them can write to it and can produce erroneous behaviour.
\end{definition}
As can be seen from the Definition \ref{04:criticalsection}, the programmer must look for such places. It can be a simple increment of a shared variable or a complex structure or object change. If these places are detected, it is necessary to perform the next step. Use Mutual Exclusion briefly stated Mutex.
\begin {definition}
  \textbf{Mutual exclusion} \---\ two threads are excluded from being in the critical section at the same time.
\end {definition}
By using a mutex, we guarantee that only one thread will access the shared resource at a time. One will have to Aquire lock whenever one wants to modify a thread or read from a shared resource. Then one modifies the source and finally release the lock. Acquire lock is an Atomic operation performed as single action and cannot be interrupted by other threads.

We know several implementations of lock, but not all of them guarantee us the Liveness property. As a reminder, the Liveness property tells us that: \emph{"A particular good thing will happen eventually"}. For example, a program never "hangs". However, they usually guarantee the Safety property, and the attentive reader would undoubtedly notice that Mutual Exclusion has a Safety property. One of the leading implementations of lock are the following:
\begin {itemize}
	\item \textbf{Reentrant lock} \---\ This type of lock can be locked unlimited times. Nevertheless, the important thing is that if we want to unlock the lock, we have to do the same number of times. The use of this type of lock can be seen, for example, in recursive functions, when we lock the lock several times and unlock the same amount of times.
	\item \textbf{Try lock} \---\ Non-blocking version of the classic lock, if the Mutex is available, it acquires the lock and returns instantly true at the same time. Otherwise, it returns false. This behaviour is beneficial if the thread can do other things than in the critical section. Therefore, it will not be blocked as a classic lock.
	\item \textbf{Read-write lock} \---\ In case more readers want to read from a shared resource can. However, once a thread is locked in ReaderLock, it is not possible to get a thread that wants to modify the value of the shared resource. This is only possible if the thread that read the value subsequently released ReaderLock for the shared resource. At this point, the thread can be locked using WriterLock, and no other thread can access it. This type of lock is intended mainly for situations where we have more threads that will read from a given shared resource and fewer threads that will write (i.e., databases).
\end{itemize}

\section{Synchronisation}
\label{04:synchronization}

The main problems posed by mutexes are, for example, \emph {busy-waiting}, deadlock, livelock or starvation.

\begin{definition}
  \textbf{Busy waiting} \---\ waiting until thread, which is in the critical section, release lock or flag. The mutual exclusion problem requires waiting, and there is no way how to avoid it.
\end{definition}
Elimination of busy-waiting is possible using another synchronisation primitive such as Semaphore or Condition variable. The Condition variable represents a queue of threads waiting for a specific event and associated with a Mutex. Using these two parts, they implement a higher abstraction called the Monitor. The Monitor is a high-level synchronisation primitive that ensures mutual exclusion while giving threads the ability to wait until an event occurs. Noteworthy is the fact that the Condition variable involves three operations:
\begin{itemize}[itemsep=1mm, parsep=0pt]
\item \textbf {Wait} \---\ If a thread locks the Mutex and then verifies the Condition variable and finds that the condition is not satisfactory, it immediately switches to the Wait state, unlocks the Mutex, and queues the wait queue. to the \emph{notify()} signal, which automatically locks the Mutex again and tests the condition variable.
\item \textbf{Signal} \---\ If a thread has finished executing, it signals with \emph{notify()} and thus wakes one thread from the \emph{Waiting} state.
\item \textbf{Broadcast} \---\ A variant of the signal operation that wakes up all threads in the queue.
\end{itemize}
Another synchronisation mechanism is a Semaphore. Sometimes also referred to as a superset of a mutex. This is because if we imagine the simplest Semaphore, we get a mutex. The main difference between a mutex and a semaphore is that the Semaphore allows access to a critical section to more than one thread simultaneously. The amount added to such a section is conditioned by the number that one initialises in the Semaphore. The basic principle is that if a thread wants to access a critical section, it must increment this number. If the number reaches zero at that moment, no other thread can access the critical section. If the thread wants to exit the critical section, it decrements the counter. Another difference between a mutex and a semaphore is that a mutex can lock and unlock the same thread, whereas a semaphore can lock and unlock a different thread.


\section{Asynchronous tasks}
\label{04:asynctaks}

Another crucial aspect of parallelisation is knowing what an asynchronous task is. It is an object that is in charge of a predetermined task. This task is performed parallel to the main thread. Imagine a situation where we have to perform several tasks. For example, create several different objects that take a certain amount of time to create. If we used the classical strategy of creating one object after another, the whole Process would take a very long time. Hence, we have another alternative; for each of these objects, we submit an asynchronous task. However, it is essential to remember that if we have only four CPUs available and want to create more tasks, for example, twelve, this will result in a situation where the other eight will have to wait until these first threads are done. Therefore, it is better to use \emph{ThreadPool} to create a new thread for each task, and in the next paragraph, we argue why.

\emph{ThreadPool} is an object that creates and manages several threads, also called worker threads. What is so interesting about \emph{ThreadPool} is that if one thread completes its task,
\emph{ThreadPool} immediately assigns a new job to the free thread.  This eliminates the creation process and thus relieving the entire load on resources. However, this is nice, but if we want to submit one asynchronous task, then in the main thread, we want the future result. Thus, we created the \emph{Future} mechanism.

\emph{Future} is another object that creates one asynchronous task. According to intuition, we could deduce that the name was given to this mechanism because we do not know the value initially, but it will be available sometime in the closing \emph{Future}. It also provides access to asynchronous operations, so most languages have the \emph{get()} method. This operation is blocking and will usually be called if one is at a point where one needs a given result from an asynchronous task. The result will be available as soon as the task is completed.

We could go on to more complex parallelisation concepts, such as partitioning, mapping, agglomeration, concurrent objects, consensus algorithms. However, these topics are not necessary to understand the following chapters. Nevertheless, if the reader has these interesting ones, we recommend reading these facts from the books \emph{An Introduction to Parallel Programming} \cite{introductionToParallelProgramming} or \emph{The Art of Multiprocessor Programming} \cite{artOfMultiprocessorProgramming}.

    \chapter{Proposal of parallel approach}
\label{04:chapter:title}

In this chapter, the author describes the overall design for parallelism the computation of the Strimzi system tests.
At first, Section~\ref{05:bottlenecks} explains the prevailing problems in the Strimzi system tests.
Then, Section~\ref{05:possibleapproaches} describes alternatives to solve these problems.
Finally, the best possible option is proposed that meets all the necessary needs.
Next, in Section~\ref{04:architecturechanges} we propose changes that have to be made, especially in the \emph{ResourceManager}, where the current algorithms for resource management are implemented and which currently do not support a thread-safe implementation.
Finally, in Section~\ref{04:methodwideparalelisation} and Section~\ref{04:classwideparalelisation} a proposal for \emph{method-wide} and \emph{class-wide} parallelisation is specified, which is described in detail with the steps that need to be done for its construction (conflicts it contains and solutions proposed using learned knowledge from previous chapters).

\section{Bottlenecks of current approach}
\label{05:bottlenecks}

As discussed in Section~\ref{02:sec:strimzisystemtests}, the time required for a given test set is extremely time-consuming.
It is easier to maintain the correctness of a program using the sequential computing model, but the benefit that parallelism offers are hard to ignore.
Nevertheless, one has to ask oneself whether it is possible and whether it worth the investment.
To answer such a question, we can use Amdahl's law, which we learned about in Section~\ref{04:amdalhlaw}.
For simplicity, assume that the unit of work will be a test case.
It will therefore be necessary to map how many tests can be parallelised.
We can find out that by analysing whether a test case contains any shared variable with other tests (i.e., shared \emph{Kafka, KafkaMirrorMaker, KafkaConnect, KafkaUser, KafkaTopic resource}).
Once it does not contain any variable, we can declare the test as parallelisable.
If a given test contains such a shared variable, it implies that it will have to run in an isolated environment.
The manual analysis we performed found that 250 tests could be run in parallel, and 115 must be isolated.
So if we apply Equation~\eqref{eqn:einstein}, (which we learned in Section~\ref{04:amdalhlaw}), the total number of tests is 365.
The parallelizable part is \emph{p = 250/365}.
The sequence part will be equal to \emph{seq = 1 - p = 115/365}.
For only four-core CPUs, we get the following acceleration~\eqref{eqn:amdalhinpractice}.
\begin{equation}
    \label{eqn:amdalhinpractice}
    S = \frac{1}{1 - \frac{250}{365} + \frac{\frac{250}{365}}{\frac{4}{1}}} =\sim 2.1x \; acceleration
    \tag{3}
\end{equation}
If we increase the number of CPUs to 8, the total acceleration will be 2.5 times, and if we scale it to 16 CPUs, the acceleration will be almost 3 times.
Consequently, if we imagine that our system tests have a total executive time of 40 hours, all tests will last approximately 13 hours with parallelisation.
Thus, with this first step, we just showed that it pays to parallelise.

Another disadvantage of the current approach is that it does not use multiple Namespaces.
In our case, for each test suite, we always have one Namespace in which we operate.
Parallelism allows us to manage multiple namespaces simultaneously while ensuring that the test cases do not overlap.
Subsequently, we create in each Namespace a Cluster Operator, again and again;
this process usually takes one minute.
The ideal approach should be that the Cluster Operator should see all Namespaces and shared them for all test suites.
Using this approach eliminates much lost time.
However, we must be aware of a particular test suite, or the test case that will require a different Cluster Operator configuration.
At that moment, we must guarantee that some label will annotate that single test case for the entire test suite to run in isolation.

The disadvantages of the current approach mentioned above may be a clear argument for why such a change is necessary.
What is also necessary to mention is the structure of the Resources classes in the Strimzi system tests.
These are classes that encapsulate both, a pre-prepared templates and, at the same time, the whole mechanism of creation.
\begin{figure}[!ht]
    \centering
    \includegraphics[scale=1]{obrazky-figures/06-proposal-of-parallel-approach/01a-azure_pipelines_with_test_results}
    \caption{Azure pipelines in form parallelism used to execute our system tests}
    \label{05:fig:azurepiplines}
\end{figure}
If we want to create a resource, we do it using \emph{KafkaResource.kafkaEphemeral(...).done()} method calls and similarly with other resources.
The correct API should propagate everything for the client writing the tests via the \emph{ResourceManager} class where a simple \emph{create()} method would be called.
Nevertheless, this fact is more a matter of architecture and not a form of the execution model.

Finally, we can discuss the last limitation for which it is necessary to make a change.
In the~\ref{02:sec:strimzisystemtests} section, we did not mention such a fact, but there is an attempt of parallelism when using the Microsoft Azure Pipelines.
On this infrastructure, we decompose our system tests into several distinctive subsets and run them as Azure separate pipelines\footnote{\textbf{Azure pipeline} \---\ one can imagine pipeline, as an Object which encapsulates multiple commands executed in order. Moreover, it is also executed as a separate process.}.
In Figure~\ref{05:fig:azurepiplines} one can see such decomposition.
The attentive reader might ask why we cannot run 40 or 100 Azure pipelines and thus reduce the total execution time of the tests.
Unfortunately, we are limited only to run six Azure Pipelines simultaneously.
By this limitation, the total set of tests takes approximately 6 hours, which is still a significant ammount of time.
Similarly, we try to reduce the time at the Jenkins pipeline when using OpenStack\footnote{\textbf{OpenStack} \---\ is a cloud computing infrastructure that manages physical machines, virtual servers or containers. At the same time, this product is one of the three most active open-sourced products globally. (\url{https://www.openstack.org/})} and Amazon Web Services infrastructure\footnote{\textbf{Amazon Web Services} \---\ also like OpenStack, is a cloud computing infrastructure that offers a myriad of services (i.e., Amazon Elastic Compute Cloud, Amazon Simple Storage Service). A very admirable attribute of this service is the availability level according to SLA (service level agreement) up to 99.9\%. (\url{https://aws.amazon.com/})}.
However, this Strimzi product must be verified for multiple configurations when running a separate Kubernetes cluster for the entire test suite.
Once we launch several such Kubernetes clusters, we are also limited by infrastructure quotas.
Overall execution time reduced can be seen in the following Figure~\ref{05:fig:jenkinspipelines}.
\begin{figure}[!ht]
    \centering
    \includegraphics[scale=1]{obrazky-figures/06-proposal-of-parallel-approach/02-jenkins-smaller}
    \caption{Jenkins pipelines in a form parallelism used to execute our system tests}
    \label{05:fig:jenkinspipelines}
\end{figure}

It is also important to mention that we are limited by the number of processes (i.e., pipelines) that always use the separate Kubernetes cluster.
On Amazon Web Services and Openstack infrastructures, we are not limited the computing resources we use.
This is the fact that we must use and thus think about how parallelisation will lead the way.
Undoubtedly, this will not be at the levels of processes, but parallelisation is possible directly in the test set (i.e., using threads) thanks to the available computational resources.
However, this decision evokes the approaches described in the next section.

\section{Possible approaches}
\label{05:possibleapproaches}

From the previous section, we could notice that any attempt to parallelise at the process level (i.e., spawn more pipelines) was impossible, especially in terms of individual infrastructures' constraints.
As a result, we have no choice but to go one level lower and try to parallelise at the test level and thus use the threads.

\subsection{Writing own testing framework}

The first and the most difficult alternative is to write a new testing framework.
One would say that this may be an old-fashion approach, but it also has its advantages.
One of the leading benefits is flexibility.
Imagine that we want to configure how many test cases and test suites we want to run simultaneously.
The natural way to do this is using Futures.
Each parallel suit is associated with its \emph{Future}, and one uses a composite future to await the completion of all of them.
We could do that by using \emph{JDK ExecutorService}\footnote{\textbf{ExecutorService} \---\ is a Java object, which provides a way to execute tasks on threads asynchronously.} and \emph{CompletableFuture}\footnote{\textbf{CompletableFuture} \---\ is a superset to Future, which we learned about at the end of Chapter \ref{03:chapter:title}. Moreover, it provides exception handling, allows us to combine CompletableFuture, and has many auxiliary methods}.
However, the problem that writing a new tool would mean writing new tests and partially rewriting them all.
Since our tests are currently designed on top of the JUnit5 platform, it is not very acceptable for us to do such a thing.

\subsection{Writing our own Junit5 Engine}

Another alternative to reduce the overall load of rewriting all tests would be to write a new JUnit5 Engine.
In this case, we would have to write the overall logic of the lifecycle test.
It would help if one remembered how we described the dependencies of the current Strimzi system tests in Section~\ref{02:subsec:strimziJunit5relation:execution}.
This dependency eliminates the worry of \emph{TestDiscovery} and \emph{TestExecution}.
Therefore, if we want to create our own \emph{TestEngine}, we have to implement our own \emph{TestDiscovery} and thus create our own implementation similar to Algorithm~\ref{02:alg:selectorresolver}.
Furthermore, we need to create our own \emph{TestExecution} mechanism.
The testing mechanism could be very similar to the previous subsection, thus using the \emph{CompletableFutures} and \emph{ExecutorService} classes that Java offers.
One may invoke the idea that this is the best approach that eliminates the discovery of all tests and the overall work on designing a new tool.
Unfortunately, it also has its disadvantages.
One of them is that if one decides to write their \emph{TestEngine}, they must realise that this eliminates all the annotation support offered by Junit5 \emph{TestEngine} (i.e., @Test, @TestFactory, @ParametrizedTest, @Isolated and @TestTemplate).
It is clear that if we write a new \emph{TestEngine}, we have to write our own annotated tests and write our own annotations.
With this knowledge, even this approach does not meet our needs.


\subsection{JUnit5 paralelisation}

The last alternative is the use of Junit5 \emph{TestEngine} parallelisation.
This, almost 3-year-old, feature of the Junit5 platform (released 3rd September in 2018) has a lot to offer.
For example, parallelisation support for running multiple test cases at one time is possible using the \emph{Java Fork / Join} framework.
This framework also includes the implementation of the \emph{ThreadPool} object, which we described in Chapter \ref{03:chapter:title}.
The overall logic utilises reusable Threads, where, for example.
Thread A completes the execution of Test 1, it will be assigned another test immediately and thus, we eliminate redundant creation of threads.
The main advantage of such an approach is that it is not necessary to rewrite a complete perform of the tests.
Moreover, it is unnecessary to implement TestDiscovery and TestExecution because JUnit5 TestEngine already offers them.
Related to this is keeping all the annotations mentioned in the previous subsection.
Another advantage is the possibility of configuration where we can enable parallelization using the following commands:
\begin{verbatim}
junit.jupiter.execution.parallel.enabled = true
junit.jupiter.execution.parallel.mode.default = same_thread
junit.jupiter.execution.parallel.mode.classes.default = concurrent
\end{verbatim}
With this setting, it is possible to run the suite test in parallel using Junit5 parallelisation.
If we change \emph{junit.jupiter.execution.parallel.mode.default = concurrent} then we let concurrent execution of test cases and test suite run simultaneously.
Another good aspect of this feature is the ability to choose the best variant of the parallelisation strategy:
\begin{itemize}[itemsep=1mm, parsep=0pt]
    \item \textbf{Fixed} \---\ \emph{ThreadPool} has a predefined number of threads to work with and can be changed in the configuration using \emph{parallel.config.fixed.parallelism}.
    \item \textbf{Dynamic} \---\ \emph{ThreadPool} has a predefined number of threads based on the calculated available processors multiplied by the number specified by \emph{parallel.config.dynamic.factor}.
    \item \textbf{Custom} \---\ possible custom implementation of the strategy.
\end{itemize}
However, this configuration does not apply to scenarios where we want to run a particular set of tests in parallel and the other sequentially.
Therefore, Junit5 also provides a possible dynamic rewrite of the configuration at build time using the @Execution annotation, which can contain two values for sequential execution (@Execution (SAME\_THREAD) ) of a test suite or test case or @Execution (CONCURRENT) for concurrent execution of class or test case.
Thanks to the mentioned annotations, we can achieve decompositions of tests that will run in parallel and sequentially.
It may be apparent to the reader that our needs will be met by using this feature of the Junit5 Engine offers.

However, another common problem with the approach we have described is that the current ResourceManagement is not ready for parallelisation.
This problem forces us to rewrite our test architecture, and with that comes the rewriting of the ResourceManager class and its Resource classes.

\section{Architecture changes}
\label{04:architecturechanges}

In this section, we will describe all the necessary changes in our system test architecture.
We start with designing thread-safe algorithms responsible for managing the resources with which the individual test cases operate.
Finally, we describe the design of individual resource classes that will use the Interface pattern\footnote{\textbf{Interface pattern} \---\ one of the most popular design patterns, which defines a set of operations and creates a contract for a class that must implement these operations.}

\subsection{Resource classes}

If we think about the whole architecture of the system tests from the section~\ref{02:sec:strimzisystemtests}, one will notice that the Resource classes contains two large pieces.
The first are management methods (i.e., create(), delete()), and the second part are predefined templates, which are then used in test cases.
Therefore, we suggest that the given parts of the code must be divided into classes, where the methods are used for management would be left in these classes.
However, predefined templates moved to the so-called \emph{Templates} classes.
With further improvements and better design, we propose to create an interface that will contain methods for resource management, and each type of Resource class will need to implement such an interface.
The given interface should consist of the following abstract methods:
\begin{itemize}[itemsep = 1mm, parsep = 0pt]
    \item \textbf {getKind()} \---\ an abstract method that will serve as a type identifier of the given resource instance,
    \item \textbf {get()} \---\ the abstract method that will serve as a single resource,
    \item \textbf {create()} \---\ the abstract method responsible for creating the resource,
    \item \textbf {delete()} \---\ the abstract method responsible for deleting a given resource,
    \item \textbf {waitForReadiness()} \---\ the abstract method, for waiting for a given resource until it is ready.
\end{itemize}
Thanks to this change, we will create a generic method at the heart of the ResourceManager class.

\subsection{ResourceManager}
\label{04:sub:sec:resourcemanager}

The most critical part of system test module is ResourceManager.
In Section~\ref{02:sec:strimzisystemtests}, we described how this class works and what exactly it contains.
To maintain the context of all resources with which the three types of stacks are currently used.
If we are in the \emph{@BeforeAll} context, then it is clear that we switch the pointer stack to the class stack.
On the other hand, before each test case, we switch to the method stack.
However, the cautious reader will realise that such a mechanism will not work in parallel executions.

As part of the change, we propose eliminating all three stacks used to maintain the context and creating a HashMap that will have the name of the test case as an identifier (key).
We create a contract for a person who creates the tests to do not equal themselves.
As a value in the given map, we will store a Stack that will store all types of resources, i.e. there will always be one stack for each test case.
Related to this section is a change in resource creation management.
We propose the following thread-safe algorithm~\ref{04:alg:creationofresource}, which eliminates the invocation of methods from individual Resource classes, but all this will be done within the ResourceManager class.
In the given algorithm, there are 3 phases:

\begin{itemize}[itemsep=1mm, parsep=0pt]
    \item \textbf{Find} \---\ finding the resource type and invoking it within the Kubernetes API,
    \item \textbf {Store and future deletion} \---\ save the resource to the stack and automatically delete it throughout the lifecycle,
    \item \textbf {Readiness check} \---\ waiting if a given resource is deployed in a Kubernetes cluster (optional).
\end{itemize}

\begin{algorithm}[H]
    \caption{Thread-safe algorithm for creation resources inside \emph{Resource manager}}
    \label{04:alg:creationofresource}
    \hspace*{\algorithmicindent} \textbf{Input: ExtensionContext context, GenericType resources}

    \begin{algorithmic}[1]
        \ForEach{$resource \in resources$}
        \State $type \gets findResourceType(resource)$
        \State $type.create(resource)$
        \State // here starts critical section
        \State $all\_resources.computeIfAbsent((test\_name), k -> new Stack<>())$
        \State $all\_resources.get((test\_name).push(deleteResource(resource)$
        \State // here ends critical section
        \If{wait for resource readiness}
            \ForEach{$resource \in resources$}
            \State $type \gets findResourceType(resource)$
            \State wait for resource readiness
            \EndForEach
        \EndIf
        \EndForEach
    \end{algorithmic}
\end{algorithm}

An essential aspect of this proposed algorithm is also the ExtensionContext, which will identify the current place of execution.
There is an ExtensionContext for each test case and it contains metadata about the test.

Another part is in case the user wants to asynchronously create ten resource instances independently of each other and then create a Barrier\footnote {\textbf{Barrier} \---\ is a mechanism in concurrency, which is used to synchronise multiple threads\/processes. Therefore, any thread\/process has to wait for all the threads\/processes in that place. Subsequently, if all threads\/processes arrive at the given place, the threads\/processes are awakened and can continue their execution} because the following verification steps require all resources.
Another thread-safe algorithm~\ref{04:alg:syncresources} does a very similar process, waiting for all resources to be created asynchronously.
The identification of which resource to wait for is within the given ExtensionContext.

\begin{algorithm}[H]
    \caption{Thread-safe algorithm for sychronising resources inside the \emph{Resource manager}}
    \label{04:alg:syncresources}
    \hspace*{\algorithmicindent} \textbf{Input: ExtensionContext context}
    \begin{algorithmic}[1]
        \State Stack<Resource> resources = resourceStack.get(context.getTestName());
        \State

        \State // sync all resources
        \ForEach{$resource \in resources$}
        \If{$resource == null$}
            \State $continue;$
        \EndIf

        \State
        \State {$type \gets findResourceType(resource)$}

        \State {$\Phi \gets getResourceWaitCondition(type)$}
        \State {$wait(resource, \Phi)$}
        \EndForEach
    \end{algorithmic}
\end{algorithm}

Finally, we have the last part, that is deleting resources from the stacks.
We propose a thread-safe algorithm~\ref{04:alg:deleteresources}, which will be used for the overall cleaning of the test environment.
Its functionality is configurable.
In the beginning, it finds out the condition of the emptiness of the map that contains all the resources.
Subsequently, if it does not contain anything, the whole execution ends.
However, if the map is not empty, deletion begins.
Once this phase is completed, everything will be deleted from the map.

\begin{algorithm}[H]
    \caption{Thread-safe algorithm for deletion of resources the inside \emph{Resource manager}}
    \label{04:alg:deleteresources}
    \hspace*{\algorithmicindent} \textbf{Input: ExtensionContext context}
    \begin{algorithmic}[1]
        \State{$\Psi \gets mapResourceEmptinessCondition(context)$}

        \If{$\Psi$}
            \State{$break;$ // everything is deleted}
        \EndIf
        \While {$!\Psi$}
            \State{// checking if some exception in scope of extension context arised}
            \State{$resources.get(context.getDisplayName()).pop().getThrowableRunner().run();$}
        \EndWhile

        \State{// remove stack from map}
        \State{$resources.remove(context.getDisplayName());$}
    \end{algorithmic}
\end{algorithm}

\section{Method wide parallelisation}
\label{04:methodwideparalelisation}

In this section, we will describe our proposal for a possible method-wide parallelisation.
Method-wide parallelisation is where each test suite will be isolated, and each test case will run in parallel, if possible.
We have already approached the condition for running the tests in parallel in Section~\ref{05:bottlenecks}.
So this is a test that does not use any shared resources.
The proposal is decomposed into several steps: which are described in the next paragraphs.

The first step is to create a unique name mechanism for all the resources that are used in the test cases.
Since these are Kubernetes system tests, the created resources do not have a random naming generated.
By randomisation, we eliminate possible conflicts that could arise in parallel execution in a given test suite.
Furthermore, random naming does not require additional synchronisation of conflicting resources because each newly created resource will have a different name.

The second step is to create Kubernetes methods that will support namespace operations.
Thanks to the Kubernetes client, which we already have in the system tests, this is possible.
However, it contains too complicated invocations of methods, and so for our purposes, it is better to encapsulate this complexity into factory methods.
These are mainly methods for communication with the Kubernetes environment (i.e., Pod, ReplicaSet, Deployment, Services, Custom Resource, Custom Resource Definition).

The third step provides a mechanism that determines which methods can be performed in parallel and which need to be isolated.
For parallel tests, we propose use the \emph{@ParallelTest} annotation.
This annotation will encapsulate the \emph{@Test} annotation, so the JUnit5 framework recognises it as a test.
It will also be necessary to add information so that the test can run in parallel.
\begin{figure}[!ht]
    \centering
    \includegraphics[scale=0.8]{obrazky-figures/06-proposal-of-parallel-approach/04-method-wide}
    \caption{The best scenario in \emph{method-wide} parallelism. \emph{n} number of threads are executed, and there is no one \emph{@IsolatedTest} in the test suite, which means that all test runs simultaneously. Note that \emph{Tc} means Test case in short.}
    \label{05:fig:methodwideparallelism}
\end{figure}
Thanks to the @Execution annotation, which will be set to the value \emph{CONCURRENT}, the test will always run in parallel.
On the other hand, tests that will require isolation will use @IsolatedTest annotation.
This annotation will be a bit more complex because it will contain not only the @Test annotation but also the read-write lock.
As a reminder from Chapter~\ref{03:chapter:title}, read-write lock consists of two types of locks.
ReaderLock allows multiple readers to read from a shared source.
However, if even one reader reads, no thread can write to the source.
If no reader reads anymore and one thread wants to write, the file will be locked using WriterLock.
Here, however, another thread cannot access until the same thread releases it.
So for the @IsolatedTest annotation, we propose using this type of lock to guarantee the system's safety property (mutual exclusion).

In Figure~\ref {05:fig:methodwideparallelism} it is possible to see the best scenario that can happen in \emph {method-wide} parallelisation.
Moreover, we must realise that if the test suite theoretically contains all \emph{@IsolatedTest}, it would be a sequential execution.
Of course, if the computer on which the tests would run contained no more than two CPUs, then it is not possible to run multiple parallel threads with each other (it is possible, but the processor would then have to make many \textbf{context switches}, which would lead to a significant decrease in performance).
Thus, the more CPUs a given computer/cluster will have available, the quickier the results.

\section{Class wide paralelisation}
\label{04:classwideparalelisation}

In this section, we will describe and suggest what steps are needed to support \emph{class-wide} parallelisation.
At first, in Section~\ref{05:sharedclusteroperator}, we describe all the necessary changes that need to be made.
Furthermore, restructuring and creating a new class for managing all possible Cluster Operator configurations.
Next, we describe the rollback mechanism needed to solve the problem with two test suites that need different configurations.
We follow up on this in Section~\ref{05:isolatedsuite}, where we solve the given problem completely.
Finally, in Section~\ref{05:parallelsuite} we propose a mechanism that determines when to execute test suites in parallel.

\subsection{Shared Cluster Operator}
\label{05:sharedclusteroperator}

This change requires multiple interventions in the test suite.
Since a new Cluster Operator is currently being created in each test suite, we must always have this Cluster Operator available in a shared context.
This is accompanied by the question of how it will be possible to obtain such a context.
In Section~\ref{04:architecturechanges}, especially in the description of the ResourceManager component, we partially described the ExtensionContext object, which serves as a test identifier thanks to a \emph{hashcode}\footnote{\textbf{Hashcode} \---\ hashcode in Java is usually an integer value that has the same number for the identical objects. However, if the objects differ in one of the instance attributes, the hashcode must have a different value. This is a known contract between a Class and its implemented \emph{int hashCode()} method.}  However, we must be aware that any ExtensionContext in either the \emph{@BeforeEach} or \emph{@BeforeAll} scopes of the code cannot be used. If we used such an ExtensionContext, the Shared Cluster Operator would be deleted after the test suite in \emph{@AfterAll} has perished. One elegant approach to solving this problem is to use the \emph{extensioncontext.getRoot()} context, which ensures that the Cluster Operator is not deleted prematurely.
Another problem is the lack of an annotation/extension that creates a shared Cluster Operator only once if multiple test suites are run.
We propose to create such an annotation \emph{@BeforeAllOnce}.
Thanks to JUnit5 and its flexibility, it will be possible to implement such a mechanism by overriding \emph{@BeforeAllCallback}.

Another significant change that needs to be made is the unification of the Cluster Operator installation.
This requires a design that encapsulates multiple configurations of the Cluster Operator and would be easy to use for the client.
The answer to this is the \emph{Builder design pattern}, which will allow the client to specify the necessary configuration it will require.
On the other hand, a person implementing this mechanism will disable parts that he does not want to make available to the user using operators' visibility (i.e., private, protected, package-protected).
This eliminates the number of factory methods currently in the project and increases the overall readability of the code.
An example of the resulting implementation and invocation for a given client might look exactly like the code shown in \ref{05:fig:clusteroperatorinstallation}.

\begin{figure}[ht!]
    \centering
    \begin{verbatim}
    // cluster operator deployment configuration
    clusterOperatorDeployment = new SetupClusterOperatorBuilder()
        (1) .withClusterOperatorName("my-cluster-operator")
        (2) .withExtensionContext(sharedExtensionContext)
        (3) .withNamespace("infrastructure-namespace")
        (4) .withWatchingNamespaces("*")
        (5) .withOperationTimeout(...)
        (6) .withReconciliationInterval(...)
        (7) .withExtraEnvVars(...)
        (8) .createInstallation()
            .runInstallation();
    \end{verbatim}
    \caption{One of the possible invocation of Cluster Operator deployment using the Builder desing pattern.}
    \label{05:fig:clusteroperatorinstallation}
\end{figure}

This may not be clear from the Figure~\ref{05:fig:clusteroperatorinstallation}, but the \emph{runInstallation()} method should encapsulate all installations such as RBAC, HELM, and BUNDLE. Each of these installations has its preparation of the environment, and therefore it is necessary to distinguish them. For clarity, we will also describe the individual parameters that we indicated in Figure~\ref{05:fig:clusteroperatorinstallation}.

\begin{enumerate}[itemsep = 1mm, parsep = 0pt]
    \item \textbf{withClusterOperatorName} \---\ will be used to specify the exact name of the Cluster Operator Deployment.
    \item \textbf{withExtensionContext} \---\ possible ExtensionContext specification for resource management.
    In this case, a shared ExtensionContext object that will ensure that the instance is not deleted prematurely.
    \item \textbf{withNamespace} \---\ specification of the Namespace name to be created for the Cluster Operator.
    In this case, the infrastructure Namespace is used.
    \item \textbf{withWatchingNamespaces} \---\ specification of the Namespaces that the Cluster Operator must observe.
    In most cases, this will be a configuration where the Cluster Operator is set to \emph{*}, which semantically means that it observes all Namespaces available in the Kubernetes cluster.
    \item \textbf{withOperationTimeout} \---\ timeout specification for Cluster Operator internal operations (ie, Kafka cluster, Kafka Mirror Maker creation).
    \item \textbf{withReconciliationInterval} \---\ specification of the control loop loop interval.
    \item \textbf{withExtraEnvVars} \---\ additional possible configurations using environment variables (i.e., Strimzi operator namespace labels or Strimzi network policy generation).
    \item \textbf{createInstallation} \---\ instance construction with pre-supplied attributes.
\end{enumerate}

The last change within the shared Cluster Operator is to create a rollback mechanism that will solve the problem if we have two test suites with different Cluster Operator configurations.
Note that it is not possible to have multiple Cluster Operator deployments, as this would overlap and at the same time disrupt the operators.
Therefore, we propose to create a rollback mechanism that will solve this problem.
The~\ref{05:alg:rollback} algorithm shows the principle of operation.
Specifically, we suggest that the algorithm be divided into two phases, where the first is to delete all currently deployed resources.
The second phase is the deployment of a new Cluster Operator with a default configuration.
\begin{algorithm}[H]
    \caption{Cluster Operator rollback algorithm}
    \label{05:alg:rollback}
    \begin{algorithmic}[1]
        \State{$// \; 1st \; phase$}
        \State{// trigger that we will again create namespace}
        \If{Environment.isHelmInstall()}
            \State{helmResource.delete();}
        \EndIf
        \If{Environment.isOlmInstall()}
            \State{olmResource.delete();}
        \EndIf
        \If{Environment.isBundleInstall()}
            \State{// clear all resources related to the extension context}
            \State{ResourceManager.getInstance().deleteResources(sharedExtensionContext));}
            \State{KubeClusterResource.getInstance().deleteNamespace(infrastructure-namespace);}
        \EndIf

        \State{$// \; 2nd \; phase$}
        \State{defaultInstance $\gets$ buildDefaultInstallation();}
        \State{deployedInstallation $\gets$ defaultInstance.runInstallation();}
        \State
        \State{return deployedInstallation;}
    \end{algorithmic}
\end{algorithm}

However, there is another problem that even this mechanism will not solve, and that is the guarantee that test suites with different Cluster Operator configurations will run in isolation.
This issue will be resolved in the following Section~\ref{05:isolatedsuite}.

\subsection{@IsolatedSuite}
\label{05:isolatedsuite}

One way to solve the problem is when we have different configurations of Cluster Operator, it is necessary to supply some form of synchronisation.
Recall \emph{@IsolatedTest} from \emph{method-wide} parallelization.
In this case, it will be no different.
We suggest using a read-write lock for the new \emph{IsolatedSuite} annotation, which will encapsulate the lock.
All the rules we described in the \emph{method-wide} parallelisation will be the same as for \emph{@IsolatedSuite}, so they will be the same semantically.
The only difference will be the fact that @IsolatedTest will be an annotation applied only to method-scope, where \emph{@IsolatedSuite} the annotation will be applied only to class-scope.

\subsection{@ParallelSuite}
\label{05:parallelsuite}

Additionally, we will have to design a mechanism for running multiple test suites in parallel.
One way how to tackle this problem is to create an annotation that overrides configuration same as \emph{@ParallelTest}
that will contain an \emph{@Execution} annotation with the value \emph{CONCURRENT} and thus guaranteeing parallel execution.
Nevertheless, we would not be able to configure method-wide parallelisation with such an approach.
So the final solution is to override configuration using system property\footnote{junit.jupiter.execution.parallel.mode.classes.default},
when we need it.
For instance, we left the default value system property for method-wide parallelisation (i.e., \emph{same\_thread}).
By contrast, we set it to \emph{concurrent} if we need to execute test suites in parallel.
At the same time, we supply metadata in the form of @ParallelSuite annotation to these classes, which can be run in parallel with other classes.
    \chapter{Implementation}
\label{05:chapter:title}

This chapter is devoted to implementing additional functionality (i.e., parallelism) into the test framework within the Strimzi project.
We selected Java programming language because the test framework is written in the given language.
Moreover, in Section~\ref{05:sec:method:wide:parallelism} we describe an implementation of the first possible level of parallelism for more minor instances of Kubernetes cluster.
Finally, for more comprehensive instances (i.e., multi-node Kubernetes clusters), we explain the implementation of even higher-level parallelism in Section~\ref{05:sec:class:wide:parallelism}.

\section{Stage \---\ method-wide parallelisation}
\label{05:sec:method:wide:parallelism}

In this section, we describe the solutions of the individual steps proposed in Section~\ref{04:methodwideparalelisation},
which were necessary to perform for adaptation to method-wide parallelisations.
We start with an explanation of how to resolve the uniqueness of test resources in Section~\ref{05:sub:sec:unique}.
Furthermore, we describe the core implementation, and the necessary reworking of test resources, as well as \emph{ResourceManager} in Section~\ref{05:sub:sec:resourcemanager}.
Next, in Section~\ref{05:sub:sec:annotations} the author present a mechanism that determines whether a given test case has to be executed in parallel or isolation.
Finally, in Section~\ref{05:sub:sec:configuration} we explain how such parallelization can be configured,
and in Section~\ref{05:sub:sec:applicability} we describe its usability within our infrastructure.

\subsection{Unique Naming for each resource\protect\footnote{https://github.com/strimzi/strimzi-kafka-operator/pull/4092}}
\label{05:sub:sec:unique}

Several sources (f.e., \emph{Kafka cluster}, \emph{KafkaConnect}, \emph{KafkaMirrorMaker}), which are used in test cases,
are necessary to work with unique names to avoid conflict.
That is why we created the class \emph{TestStorage}\footnote{TestStorage \---\ https://github.com/strimzi/strimzi-kafka-operator/pull/5446/}, which will include the necessary resources (i.e., name of the namespace, cluster, topic, producer, consumer).
All this is possible thanks to \emph{ExtensionContext}, where each test case has a different object, and therefore it can be used as a map repository.

\subsection{Resource Manager re-work\protect\footnote{https://github.com/strimzi/strimzi-kafka-operator/pull/4137}}
\label{05:sub:sec:resourcemanager}

As described in Section~\ref{04:architecturechanges}, we created \emph{Interface ResourceType <T extends HasMetadata>}.
Where \emph{T} is a generic type and can take subtypes (f.e., Kafka, KafkaBridge, KafkaMirrorMaker).
In other words, everything that contains the object \emph{HasMetadata}\footnote{HasMetadata \---\ is an interface of Kubernetes resources that contain metadata object}.
Figure~\ref{interface:resourcetype} shows the individual method signatures in the given interface.

\begin{lstlisting}[language=Java,label=interface:resourcetype,caption=Interface used across all resources,frame=tb]
public interface ResourceType<T extends HasMetadata> {
    String getKind();
    T get(String namespace, String name);
    void create(T resource);
    void delete(T resource);
    boolean waitForReadiness(T resource);
}
\end{lstlisting}
Each resource then signs a contract with the ResourceType interface in our test framework.
So, for example, for a \emph{Kafka} resource the reader can see in~\ref{interface:implementation:kafka}.
\begin{lstlisting}[language=Java,label=interface:implementation:kafka,caption=Kafka resource sings contract with ResourceType interface,frame=tb]
public class KafkaResource implements ResourceType<Kafka> {
    @Override
    public String getKind() {  return Kafka.RESOURCE_KIND;}
    @Override
    public Kafka get(String namespace, String name) {...}
    @Override
    public void create(Kafka resource) {...}
    @Override
    public void delete(Kafka resource) {...}
    @Override
    public boolean waitForReadiness(Kafka resource) {...}
    // implementation of each methods omitted for clarity
}
\end{lstlisting}

Nevertheless, the most critical part of the entire Strimzi test framework is \emph{ResourceManager}.
As described in the design~\ref{04:sub:sec:resourcemanager}, so instead of 3 stacks (i.e., pointer, class and method),
we had to adapt a solution with hash maps, which for each test case will keep each stack in which will contain the test resources.
At the same time, thanks to the proposed algorithms (\ref{04:alg:creationofresource},~\ref{04:alg:deleteresources}),
the algorithm for creating resources according to the generic type \emph{T}, finds out which method to invoke.
At the same time, a parallel algorithm for deleting individual resources from a given stack.
Finally, the~\ref{04:alg:syncresources} algorithm for synchronization of parallel generating resources is most useful
in the parallel preparation of individual resources for a given test case.
An example of such a preparation phase can be seen in~\ref{resourcemanager:sync:method}.
\begin{lstlisting}[language=Java,label=resourcemanager:sync:method,caption=Example of parallel preparation of resources,frame=tb]
// create resources in parallel (simultaneously)
resourceManager.createResource(extensionContext, false,
    KafkaTemplates.kafka().build()
    KafkaTemplates.kafkaWithMetrics().build(),
    KafkaMirrorMakerTemplates.kafkaMirrorMaker().build(),
    KafkaConnectTemplates.kafkaConnect().build(),
    KafkaClientsTemplates.kafkaClients().build()
);
// synchronize point (barrier)
resourceManager.synchronizeResources(extensionContext);
\end{lstlisting}
The overall implementation of individual algorithms (\ref{04:alg:creationofresource},\ref{04:alg:syncresources} and~\ref{04:alg:deleteresources})
can be seen in the Appendix~\ref{30:appendix:b}.

\subsection{Injection of the runtime annotations}
\label{05:sub:sec:annotations}

Another crucial part is creating a mechanism that will provide information, which test case may be executed in parallel
mode and other parallel tests or run in complete isolation.
In the~\ref{04:methodwideparalelisation} section, we propose such annotations offered by the Java language.
We implemented three types of annotations for method-wide parallelization.
The most concise annotation is \emph{@ParallelTest}, which overrides the parallelism configuration at runtime.
It is possible to see the given implementation of such an annotation on~\ref{annotation:paralleltest}.
An essential part is \emph {@Execution(ExecutionMode.CONCURRENT)}, where the semantics of this line means that the given
annotation will overwrite the given configuration from a sequential mode to parallel mode and thanks to \emph {@Retention(RUNTIME)} it will do so at runtime.

\begin{lstlisting}[language=Java,label=annotation:paralleltest,caption=Implementation of the @ParallelTest annotation,frame=tb]
@Target(ElementType.METHOD)
@Retention(RUNTIME)
@Execution(ExecutionMode.CONCURRENT)
@ResourceLock(mode = ResourceAccessMode.READ, value = "global")
@Test
public @interface ParallelTest {  }
\end{lstlisting}
Another annotation (\ref{annotation:isolatedtest}) we have implemented to be responsible for the complete isolation of is \emph{@IsolatedTest}.
At an initial glance, it is remarkably similar to the previous annotations.
However, there is one major difference when using \emph{@ResourceLock}.
When \emph {@ParallelTest} uses a read lock, \emph{@IsolatedTest} uses a read\_write lock.
The idea is that read\_write lock will completely isolate us from other tests.
Multiple \emph{@ParallelTest} will be performed at the same time, and \emph{@IsolatedTest} will wait until this lock is
released (because these two annotations share the same @ResourceLock named \emph{global}).

\begin{lstlisting}[language=Java,label=annotation:isolatedtest,caption=Implementation of the @IsolatedTest annotation,frame=tb]
@Target(ElementType.METHOD)
@Retention(RUNTIME)
@Inherited
@ResourceLock(mode = ResourceAccessMode.READ_WRITE, value = "global")
@Test
public @interface IsolatedTest {
    String value() default ""; // reason why it needs isolation
}
\end{lstlisting}
Eventually, we implemented the last annotation due to product requirements \emph{@ParallelNamespaceTest}.
This annotation is equivalent to \emph{@ParallelTest}, but there is a slight distinction.
We create an additional namespace for each such test.
Scenarios where we mainly used it is when multiple Kafka clusters are deployed for a given test or when we use \emph{KafkaMirrorMaker} (by default, we need two Kafka clusters).

\subsection{Configuration}
\label{05:sub:sec:configuration}

The method-wide parallelization configuration can be set up in several ways (a) using system properties (\ref{config:system:property}),
(b) using the junit-platform.properties configuration file (\ref{config:file}).
\begin{lstlisting}[language= Java,label=config:system:property,caption=(a) Configuration via system properties,frame=tb]
-Djunit.jupiter.execution.parallel.enabled = true
-Djunit.jupiter.execution.parallel.config.fixed.parallelism = 5
// parallel.mode.default has default value same_thread
// parallel.mode.classes.default has default value same_thread
\end{lstlisting}
In both cases, five threads will be released, where each thread will perform one test at a time, and if it finishes its work, it will move on to the next test case. This is repeated until there is no more test in the test class.
\begin{lstlisting}[language=Java,label=config:file,caption=(b) Configuration via file,frame = tb]
junit.jupiter.execution.parallel.enabled = true
junit.jupiter.execution.parallel.mode.default = same_thread
junit.jupiter.execution.parallel.mode.classes.default = same_thread
junit.jupiter.execution.parallel.config.strategy = fixed
junit.jupiter.execution.parallel.config.fixed.parallelism = 5
\end{lstlisting}

\subsection{Applicability}
\label{05:sub:sec:applicability}

Method-wide parallelization for our testing framework is most efficient for more diminutive infrastructures,
typically with parameters f.e., 25GB RAM and eight cores. We use such infrastructure as part of nightly testing.
In the circumstances, we have less power available; it is necessary to count on it that in more demanding test cases
(i.e., a test case using \emph{KafkaMirrorMaker} or several Kafka clusters), the cluster will be unstable, which will
lead to poor test results and overall test timeouts. On the other hand, in the case of more powerful infrastructure
(i.e., multi-node Kubernetes cluster), it is possible to use the following form of parallelism.

\section{Stage \---\ class-wide parallelisation}
\label{05:sec:class:wide:parallelism}

\subsection{Deployment of shared Cluster Operator across $\forall$ suites}
\subsection{Isolation of test Suites}
\subsection{SuiteThreadControlller and TestNamespaceManager}
\subsection{ForkJoinPool worker-steal algorithm limitation}
\subsection{Configuration}
.... using concurrent -Djunit \dots classes=concurrent
\subsection{Usage (big clusters i.e., 6-12 nodes machines each 25GB RAM 8 CPU)}

\section{Complications during implementation}

\chapter{Experimental evaluation}
\label{06:chapter:title}

\section{Experimental setup}
%\subsection{Azure \& Openstack clouds (podkapitola nebude vidieť...)}
\section{Results}
\section{Evaluation of the obtained results}

\chapter{Future work}
\label{07:chapter:title}


\chapter{Conclusion}
\label{08:chapter:title}

  \else
    \input{chapters/projekt-01-kapitoly-chapters}
  \fi
  
  % Kompilace po částech (viz výše, nutno odkomentovat)
  % Compilation piecewise (see above, it is necessary to uncomment it)
  %\subfile{projekt-01-uvod-introduction}
  % ...
  %\subfile{chapters/projekt-05-conclusion}


  % Pouzita literatura / Bibliography
  % ----------------------------------------------
\ifslovak
  \makeatletter
  \def\@openbib@code{\addcontentsline{toc}{chapter}{Literatúra}}
  \makeatother
  \bibliographystyle{bib-styles/Pysny/skplain}
\else
  \ifczech
    \makeatletter
    \def\@openbib@code{\addcontentsline{toc}{chapter}{Literatura}}
    \makeatother
    \bibliographystyle{bib-styles/Pysny/czplain}
  \else 
    \makeatletter
    \def\@openbib@code{\addcontentsline{toc}{chapter}{Bibliography}}
    \makeatother
    \bibliographystyle{bib-styles/Pysny/enplain}
  %  \bibliographystyle{alpha}
  \fi
\fi
  \begin{flushleft}
  \bibliography{projekt-20-literatura-bibliography}
  \end{flushleft}

  % vynechani stranky v oboustrannem rezimu
  % Skip the page in the two-sided mode
  \iftwoside
    \cleardoublepage
  \fi

  % Prilohy / Appendices
  % ---------------------------------------------
  \appendix
\ifczech
  \renewcommand{\appendixpagename}{Přílohy}
  \renewcommand{\appendixtocname}{Přílohy}
  \renewcommand{\appendixname}{Příloha}
\fi
\ifslovak
  \renewcommand{\appendixpagename}{Prílohy}
  \renewcommand{\appendixtocname}{Prílohy}
  \renewcommand{\appendixname}{Príloha}
\fi
%  \appendixpage

% vynechani stranky v oboustrannem rezimu
% Skip the page in the two-sided mode
%\iftwoside
%  \cleardoublepage
%\fi
  
\ifslovak
%  \section*{Zoznam príloh}
%  \addcontentsline{toc}{section}{Zoznam príloh}
\else
  \ifczech
%    \section*{Seznam příloh}
%    \addcontentsline{toc}{section}{Seznam příloh}
  \else
%    \section*{List of Appendices}
%    \addcontentsline{toc}{section}{List of Appendices}
  \fi
\fi
  \startcontents[chapters]
  \setlength{\parskip}{0pt} 
  % seznam příloh / list of appendices
  % \printcontents[chapters]{l}{0}{\setcounter{tocdepth}{2}}
  
  \ifODSAZ
    \setlength{\parskip}{0.5\bigskipamount}
  \else
    \setlength{\parskip}{0pt}
  \fi
  
  % vynechani stranky v oboustrannem rezimu
  \iftwoside
    \cleardoublepage
  \fi
  
  % Přílohy / Appendices
  \ifenglish
    % This file should be replaced with your file with an appendices (headings below are examples only)

% Placing of table of contents of the memory media here should be consulted with a supervisor
%\chapter{Contents of the included storage media}

%\chapter{Manual}

%\chapter{Configuration file}

%\chapter{Scheme of RelaxNG configuration file}

%\chapter{Poster}

%! Author = morsak
%! Date = 13.02.2022

\chapter{Manual}
\label{30:appendix:a}

The author assume \dots by step how to run (a) method-wide parallelisation, (b) class-wide parallelisation.
What is needed blah blah..

\chapter{Implementation details}
\label{30:appendix:b}

\begin{lstlisting}[language=Java,label=resourcemanager:complete:create:method,caption=Complete thead-safe method for parallel creation resources,frame=tb]
@SafeVarargs
public final <T extends HasMetadata> void createResource(
    ExtensionContext testContext,
    boolean waitReady, T... resources) {
    for (T resource : resources) {
        ResourceType<T> type = findResourceType(resource);
        LOGGER.info("Create/Update {} {} in namespace {}",
            resource.getKind(), resource.getMetadata().getName(),
            resource.getMetadata().getNamespace() == null ? "(not set)"
                : resource.getMetadata().getNamespace());

        // ignore test context of shared Cluster Operator
        if (testContext != BeforeAllOnce.getSharedExtensionContext()) {
            // if it is parallel namespace test we are gonna replace
            // resource a namespace
            if (StUtils.isParallelNamespaceTest(testContext)) {
                if (!Environment.isNamespaceRbacScope()) {
                    final String namespace = testContext
                        .getStore(ExtensionContext.Namespace.GLOBAL)
                        .get(Constants.NAMESPACE_KEY).toString();
                    LOGGER.info("Using Namespace: {}", namespace);
                    resource.getMetadata().setNamespace(namespace);
                }
            }
        }

        type.create(resource);

        synchronized (this) {
            STORED_RESOURCES.computeIfAbsent(testContext.getDisplayName(),
                k -> new Stack<>());
            STORED_RESOURCES.get(testContext.getDisplayName()).push(
                new ResourceItem<T>(
                    () -> deleteResource(resource),
                    resource
                ));
        }
    }

    if (waitReady) {
        for (T resource : resources) {
            ResourceType<T> type = findResourceType(resource);
            assertTrue(waitResourceCondition(resource,
                ResourceCondition.readiness(type)),
                String.format("Timed out waiting for %s %s in namespace
                %s to be ready",
                resource.getKind(),
                resource.getMetadata().getName(),
                resource.getMetadata().getNamespace()));
        }
    }
}
\end{lstlisting}

\begin{lstlisting}[language=Java,label=resourcemanager:complete:delete:method,caption=Complete thead-safe method for parallel deletion resources,frame=tb]
public void deleteResources(ExtensionContext testContext) throws Exception {
    LOGGER.info(String.join("", Collections.nCopies(76, "#")));
    if (!STORED_RESOURCES.containsKey(testContext.getDisplayName()) ||
        STORED_RESOURCES.get(testContext.getDisplayName()).isEmpty()) {
        LOGGER.info("In context {} is everything deleted.",
            testContext.getDisplayName());
    } else {
        LOGGER.info("Delete all resources for {}",
            testContext.getDisplayName());
    }

    // if stack is created for specific test suite or test case
    AtomicInteger numberOfResources =
        STORED_RESOURCES.get(testContext.getDisplayName()) != null ?
        new AtomicInteger(STORED_RESOURCES.get(
        testContext.getDisplayName()).size()) :
        // stack has no elements
        new AtomicInteger(0);
    while (STORED_RESOURCES.containsKey(testContext.getDisplayName()) &&
        numberOfResources.get() > 0) {
        STORED_RESOURCES.get(testContext.getDisplayName())
            .parallelStream().parallel().forEach(
            resourceItem -> {
                try {
                    resourceItem.getThrowableRunner().run();
                } catch (Exception e) {
                    e.printStackTrace();
                }
                numberOfResources.decrementAndGet();
            }
        );
    }
    STORED_RESOURCES.remove(testContext.getDisplayName());
    LOGGER.info(String.join("", Collections.nCopies(76, "#")));
}
\end{lstlisting}


\begin{lstlisting}[language=Java,label=resourcemanager:complete:sync:method,caption=Complete thead-safe method for synchronize resources,frame=tb]
public final <T extends HasMetadata> void synchronizeResources(
    ExtensionContext testContext) {
    Stack<ResourceItem> resources = STORED_RESOURCES.get(
        testContext.getDisplayName());

    // sync all resources
    for (ResourceItem resource : resources) {
        if (resource.getResource() == null) {
            continue;
        }
        ResourceType<T> type = findResourceType((T) resource.getResource());

        waitResourceCondition((T) resource.getResource(),
            ResourceCondition.readiness(type));
    }
}
\end{lstlisting}

\begin{lstlisting}[language=Java,label=resourcemanager:supported:resources,caption=List of supported resources inside ResourceManager,frame=tb]
private final ResourceType<?>[] resourceTypes = new ResourceType[]{
    new KafkaBridgeResource(),
    new KafkaClientsResource(),
    new KafkaConnectorResource(),
    new KafkaConnectResource(),
    new KafkaMirrorMaker2Resource(),
    new KafkaMirrorMakerResource(),
    new KafkaRebalanceResource(),
    new KafkaResource(),
    new KafkaTopicResource(),
    new KafkaUserResource(),
    new BundleResource(),
    new ClusterRoleBindingResource(),
    new DeploymentResource(),
    new JobResource(),
    new NetworkPolicyResource(),
    new RoleBindingResource(),
    new ServiceResource(),
    new ConfigMapResource(),
    new ServiceAccountResource(),
    new RoleResource(),
    new ClusterRoleResource(),
    new ClusterOperatorCustomResourceDefinition(),
    new SecretResource(),
    new ValidatingWebhookConfigurationResource()
};
\end{lstlisting}
  \else
    \input{projekt-30-prilohy-appendices}
  \fi
  
  % Kompilace po částech (viz výše, nutno odkomentovat)
  % Compilation piecewise (see above, it is necessary to uncomment it)
  %\subfile{projekt-30-prilohy-appendices}
  
\end{document}
